\documentclass[14pt,twoside]{article}

\usepackage{amsfonts,amsmath,amssymb,stmaryrd,indentfirst}
\usepackage{epsfig,graphicx,times,psfrag}
\usepackage{hyperref}
\hypersetup{colorlinks=true, urlcolor=blue, linkcolor=blue, citecolor=red}
%\usepackage{natbib}
\usepackage{pdfpages}
%\usepackage{fancyhdr} %%%%
%\pagestyle{fancy}%%%%
\pagestyle{empty}
\def\newblock{\hskip .11em plus .33em minus .07em}

\setlength\textwidth      {16.5cm}
\setlength\textheight     {22.0cm}
\setlength\oddsidemargin  {-0.3cm}
\setlength\evensidemargin {-0.3cm}

\setlength\headheight{0in} 
\setlength\topmargin{0.cm}
\setlength\headsep{1.cm}
\setlength\footskip{1.cm}
\setlength\parskip{0pt}

%%%
%%% Headers and Footers
%\lhead[\text{\small{IMPERIAL COLLEGE LONDON}}] {\text{\small{Applied Modelling and Computation Group - AMCG}}} 
%%\chead[\text{\small{AMCG}}] {\text{\small{ }}}
%\rhead[\text{\small{c.pain@imperial.ac.uk}}]{\text{\small{c.pain@imperial.ac.uk}} }
%\rfoot[\thepage]{\thepage}
%\cfoot[\text{\small{April 2005}}] {\text{\small{April 2005}}}
%\lfoot [\text{\small{http://amcg.ese.imperial.ac.uk}}]{\text{\small{http://amcg.ese.imperial.ac.uk}}}
%\renewcommand{\headrulewidth}{0.8pt}

%%%
%%% space between lines
%%%
\renewcommand {\baselinestretch}{1.5}
\pagestyle{myheadings}
\markboth{\hfill EG5066: Energy Technologies \hfill Tutorial Brief 2013/14}{EG5066: Energy Technologies \hfill Tutorial Brief 2013/14 \hfill}

\begin{document}

%\noindent{\bfseries\large MSc in Oil $\&$ Gas Engineering\hfill October, 2013}

\bigskip

\begin{center}
{\Large Nuclear Technologies: Current Issues and Future Directions}\\
{\large Nuclear $\lq$Renaissance' and Radioactive Inventory –- How to tackle the new challenges ?}\\
{\large \today}
\end{center}


\section{Introduction}

Over the past few decades, high fossil fuel prices, energy security and climate change concerns and increasing urgency about reducing greenhouse gas emissions led a large number of nations to nuclear
energy as a cleaner and more efficient source of energy. Policy debates worldwide all reflect the common recognition that nuclear energy source contributes carbon-free power and reliable base-load supply. Governments and businesses are taking action and it can well be argued that the much-talked about {\it nuclear $\lq$renaissance'} is real.  

Then, in early 2011 a powerful earthquake followed by a massive {\it tsunami} hit the north-east cost of Japan killing 15.8k people and damaging/destroying part of Japan's infra-structure. The {\it tsunami}, with waves of approximately 13 meters-high, were able to overcome the coastal defence barriers in Fukushima Daiichi NPP complex, and caused a nuclear accident with 39 (non-fatal) injuries. This was the largest nuclear accident since Chernobyl in 1986. Since then, a number of NPP worldwide were shutdown and plans to build new plants were postponed until all reasons (and consequences) of the accident are known and understood. 

There are main drawbacks associated with the development of nuclear power generation over the coming years. The main issues are:
\begin{enumerate}
\item Climate change policies support nuclear expansion –- it is a common understanding that CO$_{2}$ emissions would sharply decrease with the expansion of nuclear power plants;
\item Supportive government policy for nuclear development -– nuclear industry requires capital-intensive projects as nuclear power requires a stable and predictable investment climate. This requires efficient and stable government licensing and regulatory processes and sustainable structure for power markets;
\item Costs associated with nuclear plants will be larger for initial units therefore government subsidies are essential;
\item Demonstration of cost and performance in new designs for fast nuclear expansion; 
\item Uranium prices may rise as the nuclear power expands;
\item Security -– nuclear accidents and terrorism would have a strong impact on NPP expansion (public perception). Additionally, concerns about proliferation of nuclear materials are crucial as nuclear power expands its role in developing economies;
\item Nuclear waste does not decompose naturally under the affect of the atmosphere and therefore it can cause a number of hazards to the environment and to anyone who comes into contact with the radiation. While on-site storage of spent fuel is a technically viable solution for most nuclear plants, progress toward a permanent waste disposal plan is important for public support.
\end{enumerate}

\section{Tutorial}
You should read some of the arguments and science behind the {\it nuclear $\lq$renaissance'}. In the Tutorial Session, the students are gathered in (~10-11) groups and discuss three out of the four subjects below. This will be followed by class discussion.
\begin{enumerate}
\item Environmental policies associated with the new nuclear plants and the impact on greenhouse emissions;
\item Uranium cycle and market prices (current and future). This should include economic costs (capital and operational);
\item Fukushima Dai-ichi nuclear accidents -– impact on the nuclear industry worldwide and lessons learnt \cite{iaea_2011};
\item Nuclear waste repository: (a) projects $\&$ experiences in Europe, North America and/or Asia or (b) lessons learnt from the analogue natural Oklo nuclear reactor \cite{meshik_2005} in Gabon.
\end{enumerate}

\paragraph{Feedback:} Students should write a {\bf (max) 5-pages} report containing his/her critical view/analysis on:
\begin{enumerate}
\item Post-Fukushima economical/social climate: impact of the accident in the (i) public perception of nuclear energy, (ii) energy business and (iii) GHG emissions and carbon emission trading;  
\item Economic and environmental impact on uranium mining;
\item {\bf Review of the discussion in class} -- her/his views of the main arguments.
\end{enumerate} 
The report should be emailed to \href{mailto:jefferson.gomes@abdn.ac.uk}{jefferson.gomes@abdn.ac.uk} on (or before) Oct. 17$^{th}$. The report {\bf MUST} be in {\bf PDF} format and the {\bf Subject} of the email {\bf MUST} be {\bf EG5066: Tutorial Nuclear}. Feedback on the report will be given on Oct. 29$^{th}$.


\begin{thebibliography}{99}
\bibitem{burleson_2007} C.W. Burleson (2007) $\lq$Nuclear Afternoon –- True Stories of Atomic Disasters', Thunder’s Mouth Press.
\bibitem{clarke_1996} W.L. Clarke (1996) $\lq$The Safe Disposal of Nuclear Waste', {\it Science $\&$ Technology Review}, March, 6-16.
\bibitem{george_2007} G.R. George (2007) $\lq$Financing New Nuclear Capacity: Will the Nuclear Renaissance Be a Self-Sustaining Reaction?', {\it The Electricity Journal} 20:12-20.
\bibitem{iaea_2011} $\lq$IAEA International Fact Finding Expert Mission of the Fukushima Dai-Ichi NPP Accident following the Great East Japan Earthquake and Tsunami', \href{http://www-pub.iaea.org/MTCD/meetings/PDFplus/2011/cn200/documentation/cn200_Final-Fukushima-Mission_Report.pdf}{IAEA Report}.
\bibitem{mampaey_2006} L. Mampaey (2006) $\lq$Meeting the Energy Challenge for the Environment: The Role of Safety',{\it  Nuclear Engineering and Design} 236:1460-1463.
\bibitem{meshik_2004} A.P. Meshik, C.M. Hohenberg, O.V. Pravdivtseva (2004) $\lq$Record of Cycling Operation of the Natural Nuclear Reactor in the Oklo/Okelobondo Area in Gabon', {\it Physical Review Letters}, 93, 182302.
\bibitem{meshik_2005} A.P. Meshik, $\lq$The Workings of an Ancient Nuclear Reactor', {\it Scientific American}, October 2005.
\bibitem{pagnamenta_2009} R. Pagnamenta, $\lq$Radioactive waste to be put in £18bn hole', \href{http://www.timesonline.co.uk/tol/news/politics/article6908914.ece}{\it The Times}, November 9, 2009.
\bibitem{upson_2009} S. Upson, $\lq$Finland’s Nuclear Waste Solution', \href{http://spectrum.ieee.org/energy/nuclear/finlands-nuclear-waste-solution}{\it IEEE Spectrum – Inside Technology}, December 2009.
\bibitem{wald_2009} M.L. Wald, $\lq$Future Dim for Nuclear Waste Repository', \href{http://www.nytimes.com/2009/03/06/science/earth/06yucca.html?_r=1}{The New York Times}, March 5, 2009.
\end{thebibliography}

\clearpage

   
\end{document}
