
%\documentclass[11pts,a4paper,amsmath,amssymb,floatfix]{article}%{report}%{book}
%\documentclass[12pts,a4paper,amsmath,amssymb,floatfix]{report}%{article}%{report}%{book}
\documentclass[12pts,a4paper]{report}%{article}%{report}%{book}
%\usepackage{dcolumn,enumerate}% Align table columns on decimal point
\usepackage{enumerate,enumitem}% Align table columns on decimal point
\usepackage{bm,dpfloat}% bold math
\usepackage[pdftex,bookmarks,colorlinks=true,urlcolor=rltblue,citecolor=blue]{hyperref}
\usepackage{amsfonts,amsmath,amssymb,stmaryrd,indentfirst}
\usepackage{times,psfrag}
\usepackage{color}
\usepackage{units}
\usepackage{listings}
\usepackage{palatino}
%\usepackage{doi}
\usepackage{float}
\usepackage{perpage}
\MakeSorted{figure}
%\usepackage{pdflscape}


\definecolor{rltblue}{rgb}{0,0,0.75}


%\usepackage{natbib}
\usepackage{fancyhdr} %%%%
\pagestyle{fancy}%%%%
% with this we ensure that the chapter and section
% headings are in lowercase
%%%%\renewcommand{\chaptermark}[1]{\markboth{#1}{}}
\renewcommand{\sectionmark}[1]{\markright{\thesection\ #1}}
\fancyhf{} %delete the current section for header and footer
\fancyhead[LE,RO]{\bfseries\thepage}
\fancyhead[LO]{\bfseries\rightmark}
\fancyhead[RE]{\bfseries\leftmark}
\renewcommand{\headrulewidth}{0.5pt}
% make space for the rule
\fancypagestyle{plain}{%
\fancyhead{} %get rid of the headers on plain pages
\renewcommand{\headrulewidth}{0pt} % and the line
}

\def\newblock{\hskip .11em plus .33em minus .07em}
\usepackage{color}

%\usepackage{makeidx}
%\makeindex

\setlength\textwidth      {16.cm}
\setlength\textheight     {22.6cm}
\setlength\oddsidemargin  {-0.3cm}
\setlength\evensidemargin {0.3cm}

\setlength\headheight{14.49998pt} 
\setlength\topmargin{0.0cm}
\setlength\headsep{1.cm}
\setlength\footskip{1.cm}
\setlength\parskip{0pt}
\setlength\parindent{0pt}


%%%
%%% Headers and Footers
\lhead[] {\text{\small{EG501J -- Geothermal Energy}}}
\rhead[] {{\text{\small{Continuous Assessment (2015/16)}}}}
%\chead[] {\text{\small{Session 2012/13}}} 
\lfoot[]{Continuous Assessment}
%\cfoot[\thepage]{\thepage}
\rfoot[\text{\small{\thepage}}]{\thepage}
\renewcommand{\headrulewidth}{0.8pt}


%%%
%%% space between lines
%%%
\renewcommand{\baselinestretch}{1.5}

\newenvironment{VarDescription}[1]%
  {\begin{list}{}{\renewcommand{\makelabel}[1]{\textbf{##1:}\hfil}%
    \settowidth{\labelwidth}{\textbf{#1:}}%
    \setlength{\leftmargin}{\labelwidth}\addtolength{\leftmargin}{\labelsep}}}%
  {\end{list}}

%%%%%%%%%%%%%%%%%%%%%%%%%%%%%%%%%%%%%%%%%%%
%%%%%%                              %%%%%%%
%%%%%%      NOTATION SECTION        %%%%%%%
%%%%%%                              %%%%%%%
%%%%%%%%%%%%%%%%%%%%%%%%%%%%%%%%%%%%%%%%%%%

% Text abbreviations.
\newcommand{\ie}{{\em{i.e., }}}
\newcommand{\eg}{{\em{e.g., }}}
\newcommand{\cf}{{\em{cf., }}}
\newcommand{\wrt}{with respect to}
\newcommand{\lhs}{left hand side}
\newcommand{\rhs}{right hand side}
% Commands definining mathematical notation.

% This is for quantities which are physically vectors.
\renewcommand{\vec}[1]{{\mbox{\boldmath$#1$}}}
% Physical rank 2 tensors
\newcommand{\tensor}[1]{\overline{\overline{#1}}}
% This is for vectors formed of the value of a quantity at each node.
\newcommand{\dvec}[1]{\underline{#1}}
% This is for matrices in the discrete system.
\newcommand{\mat}[1]{\mathrm{#1}}


\DeclareMathOperator{\sgn}{sgn}
\newtheorem{thm}{Theorem}[section]
\newtheorem{lemma}[thm]{Lemma}

%\newcommand\qed{\hfill\mbox{$\Box$}}
\newcommand{\re}{{\mathrm{I}\hspace{-0.2em}\mathrm{R}}}
\newcommand{\inner}[2]{\langle#1,#2\rangle}
\renewcommand\leq{\leqslant}
\renewcommand\geq{\geqslant}
\renewcommand\le{\leqslant}
\renewcommand\ge{\geqslant}
\renewcommand\epsilon{\varepsilon}
\newcommand\eps{\varepsilon}
\renewcommand\phi{\varphi}
\newcommand{\bmF}{\vec{F}}
\newcommand{\bmphi}{\vec{\phi}}
\newcommand{\bmn}{\vec{n}}
\newcommand{\bmns}{{\textrm{\scriptsize{\boldmath $n$}}}}
\newcommand{\bmi}{\vec{i}}
\newcommand{\bmj}{\vec{j}}
\newcommand{\bmk}{\vec{k}}
\newcommand{\bmx}{\vec{x}}
\newcommand{\bmu}{\vec{u}}
\newcommand{\bmv}{\vec{v}}
\newcommand{\bmr}{\vec{r}}
\newcommand{\bma}{\vec{a}}
\newcommand{\bmg}{\vec{g}}
\newcommand{\bmU}{\vec{U}}
\newcommand{\bmI}{\vec{I}}
\newcommand{\bmq}{\vec{q}}
\newcommand{\bmT}{\vec{T}}
\newcommand{\bmM}{\vec{M}}
\newcommand{\bmtau}{\vec{\tau}}
\newcommand{\bmOmega}{\vec{\Omega}}
\newcommand{\pp}{\partial}
\newcommand{\kaptens}{\tensor{\kappa}}
\newcommand{\tautens}{\tensor{\tau}}
\newcommand{\sigtens}{\tensor{\sigma}}
\newcommand{\etens}{\tensor{\dot\epsilon}}
\newcommand{\ktens}{\tensor{k}}
\newcommand{\half}{{\textstyle \frac{1}{2}}}
\newcommand{\tote}{E}
\newcommand{\inte}{e}
\newcommand{\strt}{\dot\epsilon}
\newcommand{\modu}{|\bmu|}
% Derivatives
\renewcommand{\d}{\mathrm{d}}
\newcommand{\D}{\mathrm{D}}
\newcommand{\ddx}[2][x]{\frac{\d#2}{\d#1}}
\newcommand{\ddxx}[2][x]{\frac{\d^2#2}{\d#1^2}}
\newcommand{\ddt}[2][t]{\frac{\d#2}{\d#1}}
\newcommand{\ddtt}[2][t]{\frac{\d^2#2}{\d#1^2}}
\newcommand{\ppx}[2][x]{\frac{\partial#2}{\partial#1}}
\newcommand{\ppxx}[2][x]{\frac{\partial^2#2}{\partial#1^2}}
\newcommand{\ppt}[2][t]{\frac{\partial#2}{\partial#1}}
\newcommand{\pptt}[2][t]{\frac{\partial^2#2}{\partial#1^2}}
\newcommand{\DDx}[2][x]{\frac{\D#2}{\D#1}}
\newcommand{\DDxx}[2][x]{\frac{\D^2#2}{\D#1^2}}
\newcommand{\DDt}[2][t]{\frac{\D#2}{\D#1}}
\newcommand{\DDtt}[2][t]{\frac{\D^2#2}{\D#1^2}}
% Norms
\newcommand{\Ltwo}{\ensuremath{L_2} }
% Basis functions
\newcommand{\Qone}{\ensuremath{Q_1} }
\newcommand{\Qtwo}{\ensuremath{Q_2} }
\newcommand{\Qthree}{\ensuremath{Q_3} }
\newcommand{\QN}{\ensuremath{Q_N} }
\newcommand{\Pzero}{\ensuremath{P_0} }
\newcommand{\Pone}{\ensuremath{P_1} }
\newcommand{\Ptwo}{\ensuremath{P_2} }
\newcommand{\Pthree}{\ensuremath{P_3} }
\newcommand{\PN}{\ensuremath{P_N} }
\newcommand{\Poo}{\ensuremath{P_1P_1} }
\newcommand{\PoDGPt}{\ensuremath{P_{-1}P_2} }

\newcommand{\metric}{\tensor{M}}
\newcommand{\configureflag}[1]{\texttt{#1}}

% Units
\newcommand{\m}[1][]{\unit[#1]{m}}
\newcommand{\km}[1][]{\unit[#1]{km}}
\newcommand{\s}[1][]{\unit[#1]{s}}
\newcommand{\invs}[1][]{\unit[#1]{s}\ensuremath{^{-1}}}
\newcommand{\ms}[1][]{\unit[#1]{m\ensuremath{\,}s\ensuremath{^{-1}}}}
\newcommand{\mss}[1][]{\unit[#1]{m\ensuremath{\,}s\ensuremath{^{-2}}}}
\newcommand{\K}[1][]{\unit[#1]{K}}
\newcommand{\PSU}[1][]{\unit[#1]{PSU}}
\newcommand{\Pa}[1][]{\unit[#1]{Pa}}
\newcommand{\kg}[1][]{\unit[#1]{kg}}
\newcommand{\rads}[1][]{\unit[#1]{rad\ensuremath{\,}s\ensuremath{^{-1}}}}
\newcommand{\kgmm}[1][]{\unit[#1]{kg\ensuremath{\,}m\ensuremath{^{-2}}}}
\newcommand{\kgmmm}[1][]{\unit[#1]{kg\ensuremath{\,}m\ensuremath{^{-3}}}}
\newcommand{\Nmm}[1][]{\unit[#1]{N\ensuremath{\,}m\ensuremath{^{-2}}}}

% Dimensionless numbers
\newcommand{\dimensionless}[1]{\mathrm{#1}}
\renewcommand{\Re}{\dimensionless{Re}}
\newcommand{\Ro}{\dimensionless{Ro}}
\newcommand{\Fr}{\dimensionless{Fr}}
\newcommand{\Bu}{\dimensionless{Bu}}
\newcommand{\Ri}{\dimensionless{Ri}}
\renewcommand{\Pr}{\dimensionless{Pr}}
\newcommand{\Pe}{\dimensionless{Pe}}
\newcommand{\Ek}{\dimensionless{Ek}}
\newcommand{\Gr}{\dimensionless{Gr}}
\newcommand{\Ra}{\dimensionless{Ra}}
\newcommand{\Sh}{\dimensionless{Sh}}
\newcommand{\Sc}{\dimensionless{Sc}}


% Journals
\newcommand{\IJHMT}{{\it International Journal of Heat and Mass Transfer}}
\newcommand{\NED}{{\it Nuclear Engineering and Design}}
\newcommand{\ICHMT}{{\it International Communications in Heat and Mass Transfer}}
\newcommand{\NET}{{\it Nuclear Engineering and Technology}}
\newcommand{\HT}{{\it Heat Transfer}}   
\newcommand{\IJHT}{{\it International Journal for Heat Transfer}}

\newcommand{\frc}{\displaystyle\frac}

\newlist{ExList}{enumerate}{1}
\setlist[ExList,1]{label={\bf Example 1.} {\bf \arabic*}}

\newlist{ProbList}{enumerate}{1}
\setlist[ProbList,1]{label={\bf Problem 1.} {\bf \arabic*}}


%%%%%%%%%%%%%%%%%%%%%%%%%%%%%%%%%%%%%%%%%%%
%%%%%%                              %%%%%%%
%%%%%% END OF THE NOTATION SECTION  %%%%%%%
%%%%%%                              %%%%%%%
%%%%%%%%%%%%%%%%%%%%%%%%%%%%%%%%%%%%%%%%%%%


% Cause numbering of subsubsections. 
%\setcounter{secnumdepth}{8}
%\setcounter{tocdepth}{8}

\setcounter{secnumdepth}{4}%
\setcounter{tocdepth}{4}%

\begin{document}

\begin{center}
{\Large Continuous Assessment -- Individual Report}
\end{center}

\begin{enumerate}
%
\item Choose one of the Manuscripts in the list below and assign yourselves in the {\it MyAberdeen}'s Groups -- Deadline: September 30$^{th}$.
%
\item Written Report containing (max 10 pages):
  \begin{enumerate}
   \item Summary of the manuscript;
   \item Relevance: main motivation for the work, applications, relationship with the course, etc);
   \item Analysis and Discussion.
  \end{enumerate}
%
\item Criteria of assessment:
  \begin{enumerate}
    \item Presentation $\&$ style of writing (20$\%$): general appearance, structure and organisation, style of writing;
    \item Technical content (50$\%$): abstract (comprehensiveness and conciseness), technical description, approach, conclusions;
    \item Evidence of critical reasoning (30$\%$): methodology and analysis (relationship between the manuscript content and the course work), discussion of the main technologies and results.
  \end{enumerate}
%
\item {\bf Prepare the report as PDF file and submit it through {\it Turnitin} by Wednesday, October 21$^{st}$, 23:59 at the latest.}
%
\item Penalties for late or non-submission are as follows:
\begin{enumerate}%[(a)]
\item Up to one week late, 2 CGS points deducted;
\item Up to two weeks late, 3 CGS point deducted;
\item More than two weeks late no marks awarded.
\end{enumerate}
If late or non-submission is due to medical or other circumstances out with your control you must submit a medical certificate or other formal evidence to the PGT Office as soon as is practicable but no later than the end of Revision Week.
%
 \item {\bf Feedback will be given before November 4$^{th}$};
\end{enumerate}
%


\clearpage

\begin{center}
{\Large List of Groups / Manuscripts}
\end{center}


\begin{enumerate}[label=\bfseries Paper \arabic*:]
%
\item K. Pruess (2006) \href{http://dx.doi.org/10.1016/j.geothermics.2006.08.002}{`Enhanced geothermal systems (EGS) using CO$_{2}$ as working fluid — A novel approach for generating renewable energy with simultaneous sequestration of carbon'}, {\it Geothermics} 35:351-367.
%
\item H. Hofmann, G. Blocher, N. Borsing, N. Maronde, N. Pastrik, G. Zimmermann (2014) \href{http://dx.doi.org/10.1016/j.geothermics.2014.03.003}{`Potential for enhanced geothermal systems in low permeability limestones – stimulation strategies for the Western Malm karst (Bavaria)'}, {\it Geothermics} 51:351-367.
%
\item M. Finster, C. Clark, J. Schroeder, L. Martino (2015) \href{http://dx.doi.org/10.1016/j.rser.2015.05.059}{`Geothermal produced fluids: Characteristics, treatment technologies, and management options'}, {\it Renewable and Sustainable Energy Reviews} 50:952-966.
%
\item J-A. R. Sarr, F. Mathieu-Potvin (2015) \href{http://dx.doi.org/10.1016/j.geothermics.2014.12.002}{`Improvement of Double-Flash geothermal power plant design: A comparison of six interstage heating processes'}, {\it Geothermics} 54:82-95.
%
\item A. Franco, M. Vaccaro (2012) \href{http://dx.doi.org/10.1016/j.energy.2011.11.029}{`An integrated “Reservoir-Plant” strategy for a sustainable and efficient use of geothermal resources'}, {\it Energy} 37:299-310.
%
\item M. Zeyghami (2015) \href{http://dx.doi.org/10.1016/j.energy.2015.05.092}{`Performance analysis and binary working fluid selection of combined flash-binary geothermal cycle'}, {\it Energy} (In press).
%
\item T. Li, J. Zhu, S. Xin, W. Zhang (2014) \href{http://dx.doi.org/10.1016/j.geothermics.2014.03.009}{`A novel geothermal system combined power generation, gathering heat tracing, heating/domestic hot water and oil recovery in an oilfield'}, {\it Geothermics} 51:388-396.
%
\item A. Casasso, R. Sethi (2014) \href{http://dx.doi.org/10.1016/j.renene.2013.08.019}{`Efficiency of closed loop geothermal heat pumps: A sensitivity analysis'}, {\it Renewable Energy} 62:737-746.
%
\item R. Shortall, B. Davidsdottir, G. Axelsson (2015) \href{http://dx.doi.org/10.1016/j.esd.2015.02.004}{`Development of a sustainability assessment framework for geothermal energy projects'}, {\it Energy for Sustainable Development} 27:28-45.
%
\item R. Shortall, B. Davidsdottir, G. Axelsson (2015) \href{http://dx.doi.org/10.1016/j.rser.2015.04.175}{`A sustainability assessment framework for geothermal energy projects: Development in Iceland, New Zealand and Kenya'}, {\it Renewable and Sustainable Energy Reviews} 50:372-402.
%
\item D. Budisulistyo, S. Krumdieck (2015) \href{http://dx.doi.org/10.1016/j.enconman.2015.06.069}{`Thermodynamic and economic analysis for the pre-feasibility study of a binary geothermal power plant'}, {\it Energy Conversion and Management} 103:639-649.
%
\item J.A. Wong-Loyaa, J.A. Andaverdec, J.A. del Rioa (2015) \href{http://dx.doi.org/10.1016/j.geothermics.2015.06.002}{`Improved method for estimating static formation temperatures in geothermal and petroleum wells'}, {\it Geothermics} 57:73-83.
%
\item J. Wang, J.Wang, Y. Dai, P. Zhao (2015) \href{http://dx.doi.org/10.1016/j.geothermics.2015.01.012}{`Thermodynamic analysis and optimization of a flash-binary geothermal power generation system'}, {\it Geothermics} 55:69-77.
%
\item L. Pan, B. Freifeld, C. Doughty, S. Zakem, M. Sheu, B. Cutright, T. Terrall (2015) \href{http://dx.doi.org/10.1016/j.geothermics.2014.05.005}{`Fully coupled wellbore-reservoir modeling of geothermal heat extraction using CO$_{2}$ as the working fluid'}, {\it Geothermics} 53:100-113.
%
\item S.A. Ghoreishi-Madiseh, F. Hassani, F. Abbasy (2014) \href{http://dx.doi.org/10.1016/j.applthermaleng.2014.11.023}{`Numerical and experimental study of geothermal heat extraction from backfilled mine stopes}, {\it Applied Thermal Engineering} (In press).
%
\item C.R. Chamorro, J.L. García-Cuesta, M.E. Mondéjar, M.M. Linares (2014) \href{http://dx.doi.org/10.1016/j.renene.2013.11.065}{`An estimation of the enhanced geothermal systems potential for the Iberian Peninsula'}, {\it Renewable Energy} 66:1-14.
%


%
\end{enumerate}




%{
%  \includepdf[pages=-,fitpaper]{EG3521_OralPresentationAssessmentSheet.pdf}
%}

\end{document}
