\documentclass[14pt,twoside]{report}

\usepackage{amsfonts,amsmath,amssymb,stmaryrd,indentfirst}
\usepackage{epsfig,graphicx,times,psfrag}
\usepackage{natbib}
\usepackage{pdfpages,enumerate}% ,enumitem}
 
%%% Rotating Page
\usepackage{pdflscape}
\usepackage{afterpage}
%\usepackage{capt-of}% or use the larger `caption` package
%\usepackage{lipsum}% dummy text


\usepackage[pdftex,bookmarks,colorlinks=true,urlcolor=blue,citecolor=blue]{hyperref}

%\usepackage{fancyhdr} %%%%
%\pagestyle{fancy}%%%%
\pagestyle{empty}
\def\newblock{\hskip .11em plus .33em minus .07em}

\setlength\textwidth      {16.5cm}
\setlength\textheight     {22.0cm}
\setlength\oddsidemargin  {-0.3cm}
\setlength\evensidemargin {-0.3cm}

\setlength\headheight{0in} 
\setlength\topmargin{0.cm}
\setlength\headsep{1.cm}
\setlength\footskip{1.cm}
\setlength\parskip{0pt}

%%%
%%% space between lines
%%%
\renewcommand {\baselinestretch}{1.5}

\begin{document}

%%%
%%% \lipsum % Text before
%%%
\afterpage{%
    \clearpage% Flush earlier floats (otherwise order might not be correct)
    \thispagestyle{empty}% empty page style (?)
    \begin{landscape}% Landscape page
        \centering % Center table
        \vfill
    \end{landscape}
    \clearpage% Flush page
}
\vfill
\clearpage


%%%%%%%
%%%%%%% NERC COMMENTS: NE/N018656/1
%%%%%%%

\begin{center}
  {\Large Comments on NERC-Proposal NE/N018656/1 `Sustainable Gas Pathways for Brazil; from Microcosm to Macrocosm' by Drs Hawkes (UK) and Giudici (Brazil)}
\end{center}
\begin{enumerate}[(i)]
%
   \item The proposal is led by Dr Hawkes (ICL) and Dr Giudici (USP) and will last for 36 months. UK team also includes UCL, University of Leeds and Cardiff University;
%
   \item The proposal requested $\sim$\pounds~970.3k from NERC and $\sim$\pounds~970.6k from FAPESP, resulting in $\sim$\pounds~1.94M. An additional contribution of \pounds~500k was obtained from BG Group;
%
   \item The \underline{proposal aims to} critically assess distinct economical, solcial and engineering strategies for sustainable gas exploration and use in the Brazilian energy matrix. The proposal has a multi-dimension approach: (a) {\bf Microcosm} level (WPs 1-4), in which the potential for production of biogas and biomethane associated with bioethanol industry is exploited, and (b) {\bf Macroscosm} level (WPs 5-8)that aims to link activities of the Microscopic level into the national level (energy, economic and social vectors) towards 2050 energy goals;
% 
   \item As such, this proposal is truly multi-disciplinary with 6 work-packages:
      \begin{description}
           \item[WP1:] Optimised distillary modelling (USP)
           \item[WP2:] Sugarcane biogas production (USP) 
           \item[WP3:] Optimised energy management in sugarcane industry: an integrated approach combining economic and environmental factors (USP)
           \item[WP4:] Socio-economics of biomethane pathways (UCL)
           \item[WP5:] Gas infrastructure in Brazil (ICL)
           \item[WP6:] Gas partnering with renewable electricity (ICL)
           \item[WP7:] Gas pathways for Brazil (ICL)
           \item[WP8:] Ecological implications of natural gas expanion for Brazil (ULeeds)
      \end{description}
%
   \item {\it Risks and mitigation strategy} are realistic and accurate;
%
   \item Reviewers' scores:
       \begin{center}
          \begin{tabular}{c c | c c c c }
              \hline
               {\bf R} & {\bf Reviewer} & {\bf Expertise} & {\bf Excellence}  & {\bf Scientific Objectives} & {\bf Non-Scientific Objectives} \\ 
              \hline
                 1     &  213805105    &  Medium         &      6            &          6                  &         5                       \\
             2$\star$  &  168184013    &  Medium         &      6            &          5                  &         6                       \\
                 3     &  186054007    &  High           &      5            &          5                  &         5                       \\
              \hline
          \end{tabular}
       \end{center}
%
   \item Reviewers' main comments:
       \begin{description}
%
          \item[R1] Multidisciplinary proposal: from engineering to socio-economical and ecological sciences;
          \item[R1 $\&$ R3] Time allocated for the PI is too small and management role of the CoI is too not clear:
            \begin{description}
               \item[Rebut:] Balanced load between UK's and Brazil's PIs;
               \item[Rebut:] Prof Brandon (time funded by BG) will act as mentor to UK's PI;
               \item[Rebut:] PI is Deputy Director of the SGI (Sustainable Gas Institute) with strong synergy of the activities;
            \end{description}
          \item[R2] Strong comments on the bio-derived gas ({\it Microcosm}) focus on the proposal. Reviewer believed that the case of support was not strong enough, with issues on:
            \begin{enumerate}
                \item potential of raw material (bagasse);
                \item choice of conversion technology (bioethanol waste $\rightarrow$ methane): {\it `The exclusive application of bio-processes for methane is surprising given the attention currently being paid to thermally based systems in the UK and Europe which offer potential for technical and economically efficient biomethane production. This is especially important give the long time horizons proposed.'} (see paragraph $\star$). Also:
                   \begin{itemize}
                       \item {\it `The concept of replacing bioethanol by biomethane seems implausible given the high export value of ethanol and its importance as a biofuel. Biomethane needs extensive upgrading to either compress it or liquefy it into a transport fuel and it is not clear if this aspect is properly considered. } 
                          \begin{description}
                            \item[Rebut:] `We intend to investigate how a biogas/bio-methane industry could exist alongside the ethanol industry.'
                          \end{description}
                       \item {\it The advocacy of dry fermentation is not well justified in the context of development of a national biomethane programme, especially with a focus on lab scale experiments;}
                          \begin{description}
                            \item[Rebut:] Technological choice of the investigators that believe this is a potential game-changer.
                          \end{description}
                       \item {\it Microgrids are proposed although it is not clear what they will provide and where or to whom.}
                          \begin{description}
                            \item[Rebut:] \textcolor{red}{Not addressed}.
                          \end{description}
                       \item {\it Substitution of diesel by methane is not just a simple case of replacement but has substantial infrastructural, technical and economic consequences which appear to be minimised.}
                          \begin{description}
                            \item[Rebut:] \textcolor{red}{Not addressed}.
                          \end{description}
                   \end{itemize}
            \end{enumerate}
            \begin{description}
               \item[Rebut:] Figure (7Mt) of bagasse is the current waste with ethanol production. But this can increase to 51Mt and become a potential raw material for biomethane production; 
            \end{description}
          \item[R3] argued that it is not clear that the {\it Macrocosm} will have a broad view instead of focusing on bio-conversion elements;
            \begin{description}
               \item[Rebut:] Investigators clarified that this was covered, i.e., the {\it Macrocosm} has broad view on the gas sustainability technologies and how it impact in the society;
            \end{description}
          \item[R1] argued that {\bf WP4} does not accurately address the socio-economics aspects of biomethane exploration.
            \begin{description}
               \item[Rebut:] Investigators clarified the meaning of {\it assemblage} and how this methodology will be used to help integrate {\it Micro-} and {\it Macro}-scales;
            \end{description}
%          
       \end{description}
%
     \item Reviewers' comments on Fit to Scheme and Pathways to Impact:
%

       \begin{description}
          \item[Fit to Scheme -- Scientific Objectives:] {\bf R1} acknowledges the scientific fit2scheme of the proposal, however, {\bf R2} and {\bf R3} are less enthusiastic. {\bf R3} comments that the proposal only address ONE sustainable gas pathway. This was acknowledged by the investigators as one technology was chosen for the micro-scale whereas several socio-economical approaches for pathways are addressed in the macro-scale. {\bf R2} is however unhappy with the approach taken. His/her comments were very strong and although (most of the arguments) rebutted by the investigators.
%
          \item[Fit to Scheme -- Non-Scientific Objectives:] Positive comments from all reviewers. The allocated time for the PI (and Co-Is) was raised again here by {\bf R1} and {\bf R2}. 
%
          \item[Pathways to Impact:] {\bf R3} was very positive with the actions towards P2I, however {\bf R1} highlighted that the 2 institutes SGI (Sustainable Gas Institute) and GIC (Gas Innovation Institute) are still in their childhood and are not well-established yet. Her/his main concern is the fact that there is no (current) engagement with the {\it `development and climate change communities'}. {\bf R2} was less positive (as already mentioned in her/his comments -re the proposal). 
            \begin{description}
               \item[Rebut:] Investigators clarified the infra-structure they put on place (management board groups) for the proposal and highlighted the production of a White Paper at the end of the project.
            \end{description}
       \end{description}
     
%
\end{enumerate}

     \begin{center}
       {\large Overall comments:}
     \end{center}

       \begin{enumerate}[1.]
          \item \textcolor{red}{Comments of {\bf R2} (168184013) \underline{does not seem} to be in-line with the overall assessment.}
%
          \item \textcolor{red}{Deliverable ?}
%
          \item {\bf Pathways to Impact:}
              \begin{enumerate}[(a)]
                 \item A good {\it Pathways to Impact} but strongly based on the alleged infra-structure of the existing SGI and GIC;
%
                 \item Exchange of research students, early-career researchers and senior researchers across the Atlantic has the potential to produce a strong impact on both communities, in particular considering that there are 4 UK universities (ICL, UCL, ULeeds and CardiffU) engaged in the project
                 \item Website and White paper are good ways for dissemination within industrial and academic stakeholders. But what about the communities in Brazil and UK? Are there anything in place for it? 
              \end{enumerate}
%
          \item {\bf Justification of Resources:} Seems reasonable. Brazilian's partners seems to be spending $\sim$12$\%$ of the FAPESP budget in a single experimental kit.
%       
          \item Suggestions for  Marks:
              \begin{enumerate}[(a)]
                 \item {\bf Research Excellence (0-10):} It is a good proposal with several interesting elements of multi-disciplinary integration from engineering to policy-making with potential strong impact on the academic communities and industrial stakeholders. The proposal is timely as researcher from Brazil and UK have an outsanding track record on their fields relevant for this Call and the collaboration may produce scientific papers and reports of high-impact. However it seems that the proposal fails to highlight the innovation that would be expected in such a call. \textcolor{red}{7}
                 \item {\bf Fit2Scheme Scientific Objectives (0-6):} The scientific aspects of the proposal fits the Call's requirement as highlighted by {\bf R1} and {\bf R3}. However, {\bf R2} raised a few issues that need to be taken into account, in which some of them were not addressed by the investigators' responses. \textcolor{red}{5}
                 \item {\bf Pathways to Impact (Acceptable/Unacceptable):} \textcolor{red}{Acceptable}.
              \end{enumerate}
%
       \end{enumerate}

\medskip

%%%%%%%
%%%%%%%
%%%%%%%


\clearpage

%%%%%%%
%%%%%%% NERC COMMENTS: NE/N018656/1
%%%%%%%

\begin{center}
  {\Large Comments on NERC-Proposal NE/N020294/1 `Novel Routes to Added Value Chemicals and Fuels from CO2 (CO2FUEL)' by Drs Dupont (UK) and Rossi (Brazil)}
\end{center}

\begin{enumerate}[(i)]
%
   \item The proposal is led by Prof Dupont (UNott) and Dr Rossi (USP) and will last for 36 months;
%
   \item The proposal requested $\sim$\pounds~891.7k from NERC and $\sim$\pounds~544.8k from FAPESP, resulting in $\sim$\pounds~1.44M;
%
   \item The \underline{proposal aims to} study integrated process for catalytic conversion of CO$_{2}$ to biofuels using ionic liquid-stabilised metal nanoparticles (IL-MNPs);
% 
   \item As such, the proposal is divided into 6 work-packages:
      \begin{description}
           \item[WP1:] Synthesis of IL and supported materias (UNott and USP)
           \item[WP2:] Catalyst Synthesis (UNott) 
           \item[WP3:] Characterisation of Catalytic Materials (USP)
           \item[WP4:] CO$_{2}$ capture and conversion mechanistic studies (USP)
           \item[WP5:] Sustainable H$_{2}$ generation (?)
           \item[WP6:] Process Integration (UNott)
      \end{description}
%
   \item {\it Risks and mitigation strategy} are realistic and accurate;

   \item {\it Deliverables} seems accurate and realistic;
%
   \item Reviewers' scores:
       \begin{center}
          \begin{tabular}{c c | c c c c }
              \hline
               {\bf R} & {\bf Reviewer} & {\bf Expertise} & {\bf Excellence}  & {\bf Scientific Objectives} & {\bf Non-Scientific Objectives} \\ 
              \hline
                 1     &  037509021    &  High           &      6            &          5                  &         6                       \\
                 2     &  117269705    &  Medium         &      6            &          6                  &         6                       \\
                 3     &  198643903    &  High           &      4            &          3                  &         3                       \\
              \hline
          \end{tabular}
       \end{center}
%
   \item Reviewers' main comments on {\it Excellence}:
       \begin{description}
%         
          \item[R1 and R2] are very supportive and raised minor issues that the investigators acknowledged and rebutted. Comments were about how the proposal will address reduction of poverty and selectivity of Fischer-Tropsch process and scale-up of the whole technology.
%
         \item[R3] comments were stronger and pointed out a number of issues:
             \begin{enumerate}
                \item Track-record of CoIs relevant to FT and RWGS catalysis: not addressed;
                \item Whole system approach: good overall response, although the investigators did not fully address the high temperature issue;
                \item {\it `The applicants do not address the most significant risk, namely that a low temperature RWGS catalyst will not be found, and that elevated reaction temperatures are still required which mean IL based systems cannot be employed due to
instability.} This comment was not fully addressed and it may be a risk for the whole project.
             \end{enumerate} 


       \end{description}
%
     \item Reviewers' comments on Fit to Scheme and Pathways to Impact:
%

       \begin{description}
          \item[Fit to Scheme -- Scientific Objectives:] {\bf R1} acknowledges the scientific fit2scheme of the proposal, bu highlights the lack of emphasis on poverty reduction, which was acknowledge by the investigators. {\bf R2} recognised the scientific value of the proposal. {\bf R3} are much less enthusiastic and highlighted that the proposal has engaged with stakeholders on the gas value chain. The investigators highlighted that CO$_{2}$ can be potentially sourced from biomass power stations but did not address the lack of industrial engagement. Maybe the most important comment from {\bf R3} was the lack of multi-disciplinary aspects in the proposal. The two research groups has complemented chemical (and physics) skills -- i.e., `focus is more on fundamental science, which while valuable is not addressing the overall objectives of the call'.
%
          \item[Fit to Scheme -- Non-Scientific Objectives:] Positive comments from all reviewers. They also highlighted specialised offices on PI and strong collaboration prior to the proposal
%
          \item[Pathways to Impact:] Positive comments from all reviewers. They highlighted {\it public awareness} aspect (to academic and lay audience) that will be raised by the investigators. The high-quality of the investigators were acknowledge by the reviewers.
       \end{description}
     
%
\end{enumerate}

     \begin{center}
       {\large Overall comments:}
     \end{center}

       \begin{enumerate}[1.]
          \item \textcolor{red}{Comments of {\bf R3} (198643903) \underline{were extremely negative} in the assessment of the proposal but somehow milder in the Fit2Schemes aspects. However this was not projected in the overall assessment.}
          \item {\bf Pathways to Impact:}
              \begin{enumerate}[(a)]
                 \item It is an extremely well-written P2I;
                 \item Exchange of research students, early-career researchers and senior researchers across the Atlantic has the potential to produce a strong impact on both communities. An extra PhD student and PDRA will be funded by UNott
                 \item Engagement of the general public;
                 \item Work closely with IP offices
              \end{enumerate}
%
          \item {\bf Justification of Resources:} Seems reasonable. Brazilian's partners seems to be spending $\sim$40$\%$ of the FAPESP budget in a equipment. Low budget for travelling is a concern.
%       
          \item Suggestions for  Marks:
              \begin{enumerate}[(a)]
                 \item {\bf Research Excellence (0-10):} It is a good proposal with several interesting elements of multi-disciplinary integration from engineering to policy-making with potential strong impact on the academic communities and industrial stakeholders. The proposal is timely as researcher from Brazil and UK have an outsanding track record on their fields relevant for this Call and the collaboration may produce scientific papers and reports of high-impact. The technology seems to be an issue, according to {\bf R3} although (s)he acknowledged the overall technical quality of the proposal. \textcolor{red}{8}
                 \item {\bf Fit2Scheme Scientific Objectives (0-6):} The scientific aspects of the proposal fits the Call's requirement as highlighted by {\bf R1} and {\bf R3}. However, {\bf R3} raised a few issues that need to be taken into account, in which some of them were not addressed by the investigators' responses. \textcolor{red}{5}
                 \item {\bf Pathways to Impact (Acceptable/Unacceptable):} \textcolor{red}{Acceptable}.
              \end{enumerate}
%
       \end{enumerate}

%%%%%%%
%%%%%%%
%%%%%%%


\clearpage

%%%%%%%
%%%%%%%
%%%%%%%




\begin{center}
  {\Large Review of the Manuscript BMSE-D-15-01072 `Numerical Study to Investigate the Effect of Inlet Velocity on Thermal-Fluid Phenomena in the Supercritical CO$_{2}$ cooled Pebble Bed Reactor' by Latifi and Setayeshi}
\end{center}

\medskip

The manuscript describes thermal fluid dynamic simulations in a pebble-bed reactor using a commercial CFD software. The manuscript briefly described the main conservative equations along with a short summary of the solution methods used. The aim of the paper is to `investigate the effect of inlet fluid velocity' in the temperature field under steady-state conditions.

The manuscript is relatively well-written with a small number of typos and unrevised sentences. A few sentences are confusing and disconnected with no clear objectives.  However, the manuscript is comprehensive although it introduced a very superficial literature review on the main subject areas (i.e., PBMR, thermalised reactors, VHTR, turbulence models, heat transfer models for granular flows, CFD methods etc). In fact, as the manuscript mainly focuses on simulations' results, I would expect a short literature review on any (if not all) of these subjects. There is a good number of manuscripts on the nuclear and non-nuclear open literature on both internal (single- and multiphase) flows and heat transfer in PBMR (e.g., see Boer {\it et al.}, 2010, Nuclear Engineering and Design 240:2384-2391; Jiang {\it et al.}, 2006, Annals Nuclear Engineering 33:1039-1057; Nouri-Borujerdi and Ghomsheh, 2015, Annals Nuclear Engineering 78:485-492; Pavlidis and Lathouwers, 2013, Nuclear Engineering and Design 264:161-167; etc).

The manuscript also fails to clearly state the novelty of the work -- as already mentioned, pebble-bed (nuclear) reactors have been widely studied, and it would interesting if the authors could highlight their contribution to science and/or technology on fluid dynamics and/or heat transfer involving flows in pebble beds.A few comments:
\begin{enumerate}[(a)] 
%
\item  The main objective of the paper is to investigate steady-state heat transfer under prescribed fluid velocity boundary conditions. However, the studied problem was not fully described, e.g., (a) is the source term, $S_{h}$ (Eqn. 5) a constant? (b) Were wall-solid and wall-fluid heat transfer mechanisms considered in the problem? If so what were the boundary conditions used in this case? Several authors reported that the temperature profile of the pebbles are nearly parabolic -- here it is nearly flat, Fig 5);
%
\item Why did the authors ignored the time-term in Eqn. 4?
%
\item Why was supercritical CO$_{2}$ used as cooling fluid? How is the compressibility of this fluid model in this work?
%
\item Mesh independence is assessed in Fig. 3, however the domain is 4.5 m long and here mesh independence was investigated over 11 m? Is this correct? Also, mesh independence (as any statistical quantity from discrete methods) is often investigated using scalar- or vector- fields that are primarily solved in the system of equations, e.g., velocity, pressure or volume fraction (for multiphase flows). Why did the author choose temperature?
%
\item Was the model used/developed by the authors previously validated against analytical solutions, experimental data, or other computational models (cross-code validation)? Some of the results does not seem correct, e.g., constant radial temperature (Fig 5), linear behaviour of temperature (Fig. 4), etc. 
%
\item Finally, was the numerical experiment conducted in annulus geometry?
%
\end{enumerate}
I have 2 main concerns about this manuscript that the authors should address,
\begin{enumerate}[(i)]
\item An in-depth analysis of the numerical solutions presented;;
\item The authors must clearly state the novelty that the manuscript addresses.
\end{enumerate}

In summary, the subject of the manuscript is very relevant, however the paper still needs to be improved before being considered for publication. The results, although interesting (but poorly discussed), does not bring much novelty to the field -- or at least the authors failed to explicitly demonstrate it. 


%%%
%%% Appendix
%%%
{
  \includepdf[pages=-,fitpaper, angle=0]{./Scan/Scan_BMSE_D_01072.pdf}}



\clearpage

%%%%%%%
%%%%%%%
%%%%%%%


\begin{center}
  {\Large Review of the Manuscript CAM-D-15-02454 `Mixed Volume Element Combined with Characteristic Mixed Finite Volume Element Method for Oil-Water Two-Phase Displacement Problem' by Yuan {\it et al.}}
\end{center}

\medskip

The manuscript describes a hybrid FEM-FVM-Characteristic method to solve the elliptic coupled mass balance and force balanced equations representing the (Darcy) flows in porous media. The authors undertook a comprehensive literature review of the main fluid methods with focus on mixed FEM.

The manuscript is relatively well-written with a small number of typos and unrevised sentences. A few sentences are confusing and disconnected with no clear objectives. However, the manuscript is comprehensive and introduces an insightful review of the main fundamentals on FEM-based methods for porous media flows. 

The manuscript however fails to highlight:
  \begin{itemize}
    \item Aims and objectives of the work
    \item Novelty.
  \end{itemize}
A few comments:
\begin{enumerate}[(a)] 
%
\item {\it Abstract} and {\it Section 1:} authors should include aims/objectives of the work;
%
\item {\it Section 2:} Are $\alpha_{i}$ (with $i$ = 1,2) arbitrary constants? Also (page 5), $D_{j}$ and $d_{j}$ were used before being defined. I am not if the former is a component of the diffusion coefficient (Eqn. 1.2) or a linear-operator.
%
\item {\it Section 5:} Validation undertook for the method introduced in {\it Section 3} seems very limited. And how does it compare (accuracy) with Arbogast $\&$ Wheeler? Analysis of the solutions in this section were superficial.
%
\end{enumerate}
I have 3 main concerns about this manuscript that the authors should address,
\begin{enumerate}[(i)]
\item An in-depth analysis of the numerical solutions presented in Figs. 2-9;
\item The authors must clearly state the novelty that the manuscript addresses.
\end{enumerate}

In summary, the subject of the manuscript is very relevant, however the paper still needs to be improved before being considered for publication. The results, although interesting (but poorly discussed), does not bring much novelty to the field -- or at least the authors failed to explicitly demonstrate it. 

\clearpage

%%%%%%%
%%%%%%%
%%%%%%%


\begin{center}
  {\Large Review of the Manuscript POWTEC-D-15-01601 `Discrete Element Modelling of Sediment Falling in Water' by Wang $\&$ Tan}
\end{center}

\medskip

The manuscript describes a set of numerical simulations of granular particles falling in fluid at rest (thus no imposed fluid velocities into the particles). The authors undertook a brief, but comprehensive, literature review of the main computational tecnologies on granular flows with focus on continuum (mainly CFD) and discrete models (mainly DEM) currently used. The main aim of the paper is to investigate (and validate) momentum transfer forces and formulations (Table 2) in the DEM framework. Simulations were performed usin g the open-source LAMMPS model and compared against lab experiments.

The manuscript is relatively well-written with a small number of typos and unrevised sentences. A few sentences are confusing and disconnected with no clear objectives. However, the manuscript is comprehensive and introduces an insightful (but superficial) review of fluid-solid mometum transfer models. The manuscript briefly reviewed the main equations for drag and added mass forces, and briefly summarises the Stokeslet disturbance formulation. 

There were other publications that extensively investigated the impact of drag and added mass forces in granular flows and, in particular, in fluidisation in both continuum and discrete models. Momentum transfer models based on solid concentration (volume fraction) and the impact of particles' motion on the surrounding fluid flows have been largely studied by the granular CFD community. A few comments:
\begin{enumerate}[(a)] 
%
\item Page 2: `(...) and complex fluid systems, that CFD results are generally non-disputed (...)': Not really, CFD, as any model, relies on extensive software and model quality assurance (verification and validation); 
%
\item Page 7: `(...) which are related to the particles material properties (see Table 1)': I believe the authors have not included this table;
%
\item Second paragraph of page 16: The authors described the numerical solutions (Fig. 2) but no analysis was conducted;
%
\item First paragraph of page 18: Why do the models behave in such way? 
%
\end{enumerate}
I have 3 main concerns about this manuscript that the authors should address,
\begin{enumerate}[(i)]
\item An in-depth analysis of the numerical solutions presented in Figs. 2-5;
\item Comparison against experimental data is very superficial and lacks a proper statistical analysis;
\item The authors must clearly state the novelty that the manuscript addresses.
\end{enumerate}

In summary, the subject of the manuscript is very relevant, however the paper still needs to be improved before being considered for publication. The results, although interesting (but poorly discussed), does not bring much novelty to the field -- or at least the authors failed to explicitly demonstrate it. 

\clearpage


%%%%%%%
%%%%%%%
%%%%%%%


\begin{center}
  {\Large Review of the Manuscript PNUCENE-D-15-00347 `Evaluation the effects of steam state equation, spray and multi-cell subdivisions on Containment pressurization modelling in a LB-LOCA' by Noori-Kalkhoran $\&$ Matajy-Kojouri}
\end{center}

\medskip

The manuscript summarises a global system methodology to assess a few safety parameters during a LB-LOCA event. It briefly describes two strategies to model heat flux and pressure oscillations in reactor containments: (a) single (or one-volume) cell and, (b) multi-cell models. Both methods assume mass conservation in a cell (or across cells) and attempt to calculate heat flux and phase change with imposed initial conditions (severe accident condition).

The manuscript is relatively well-written with a number of typos and unrevised sentences (see the attached scanned document). A few sentences are confusing and disconnected with no clear objectives. However, the manuscript is comprehensive and introduces an insightful (but superficial) review of the methods used to model heat and mass flux (thermo-hydraulic) in reactor containments.  I would expect a short literature review on the main subject areas of the paper (e.g., transient heat transfer with phase change, volume system models, etc). The manuscript also briefly reviewed the main equations used to model these phenomena. In fact, my main concern is that the equations were not fully explained, i.e., main constraints and range of applicability. And also, it is not clear if the authors developed the models using these equations or if they used an industry-standard system code/software. A few comments:
\begin{enumerate}[(a)] 
%
\item Several terms use different symbols throughout the manuscript, making the reading more difficult, e.g., specific enthalpy ($i$, $h$ and $H$), mass flow ($\dot{M}$ and $G$), spray (subscripts $sp$ and $\text{\it spray}$). Also some indices have changed throughout the sections, e.g., $k$ and $K$ from Eqn. 42 onwards;
%
\item The authors should consider adding more descriptive (and self-contained) figure captions;
%
\item Some terms were used before being defined, e.g., IRIS, FSAR, LB-LOCA etc;
%
\item Section 2.1 was very confusing and not well explained -re the validity of the equations;
%
\item Section 2.2:
   \begin{enumerate}[(i)]
      \item Eqn. 9 should read as,
        \begin{displaymath} 
           M_{D}\left(t+\Delta t\right) = \int\limits_{t}^{t+\Delta t} \left(\dot{M}_{DB}-\dot{M}_{SC}-\dot{M}_{WC}\right)dt + M_{D}(t)
        \end{displaymath}
      \item The term `{\it flash}' in the first line of this section (also throughout the text), does it refer to isothermal (or isenthalpic) flash or just to fluid flow into/out of the containment?
      \item In Eqn. 12, how are $P_{\text{max}}$ and $P_{\text{min}}$ calculated/obtained? 
   \end{enumerate}
%
\item Equations of state for the water/steam-air system are briefly described in Section 2.3.1. Why do the authors assumed that air should behave as an ideal gas? Properties of saturated water and superheated steam were obtained from water/steam thermodynamic tables. Are these tabulated or calculated `{\it on the fly}' (i.e., online)?
%
\item In Section 3.1, it seems that the author assume all fluids incompressible (Eqn 27), why?
%
\item Section 5:
   \begin{enumerate}[(i)]
      \item Could the authors explain the DECL accident scenario? I guess this would be useful for the PNE readers. Why was this accident scenario chosen?
      \item Results were shown in Figs 9-18 (with and without spray). However no reason was given for the best performance of the multi-cells system (in comparison with the single-cell). Also, why is there large discrepancy (pressure and temperature fields) between FSAR and the calculated models in the beginning of the transient?  
   \end{enumerate}   
%
\end{enumerate}

In summary, the subject of the manuscript is very relevant for nuclear technology community, however the paper still needs to be largely improved before being considered for publication. The results, although interesting (but poorly described and discussed), does not bring much novelty to the field -- or at least the authors failed to explicitly demonstrate it. 



In the scanned copy,
\begin{itemize}
\item PE: Poor English. The authors should consider revise the sentence(s);
\item SC: Sentence(s) is/are very confusing and do(es) not make any/much sense. The authors should revise the sentence(s).
\end{itemize}


%%%
%%% Appendix
%%%
{
  \includepdf[pages=-,fitpaper, angle=0]{./Scan/Scan_PNUCENE-D-15-00347.pdf}

%%%
%%% \lipsum % Text before
%%%
\afterpage{%
    \clearpage% Flush earlier floats (otherwise order might not be correct)
    \thispagestyle{empty}% empty page style (?)
    \begin{landscape}% Landscape page
        \centering % Center table
           \Huge{2015}
        \vfill
    \end{landscape}
    \clearpage% Flush page
}
\vfill
\clearpage

%%%%%%%
%%%%%%%
%%%%%%%


\begin{center}
  {\Large Review of the Manuscript AMC-D-15-01974 `A New Metropolis Optimization Method, the Cross-Section Particle Collision Algorithm' by Sacco $\&$ Henderson}
\end{center}

\medskip

The manuscript introduces a modified PCA optimisation algorithm, based on the phase-space search-acceptance strategy within the scattering stage. The performance of the CSPCA algorithm was assessed through 5 optimisation test-case problems.

 
The manuscript is well-written with a small number of typos and unrevised sentences (see the attached scanned document). A few sentences are confusing and disconnected with no clear objectives. 

The method and benchmarks introduced in the manuscript are very challenging, however reading the manuscript is difficult to identify the authors' original contribution. It is clear that CSPCA is a more complex algorithm than the original PCA -- and also more efficient for the set of non-linear benchmark test-cases. The weighted acceptance Monte Carlo-based criteria has some resemblance with the cross-section generation, so how important the scattering stage is within the PCA family of methods? The overall performance of the CSPCA, highlighted in Tables 1-4, does not seem to differ that much from the traditional PCA, if we consider both SR and average NFE. Why? May it be a problem in the scattering stage implementation (maybe in the way the probabilistic function is defined)?  

%%%
%%% Appendix
%%%
{
  \includepdf[pages=-,fitpaper, angle=0]{./Scan/Scan_AMC-D-15-01974_Review.pdf}


%%%
%%% \lipsum % Text before
%%%
\afterpage{%
    \clearpage% Flush earlier floats (otherwise order might not be correct)
    \thispagestyle{empty}% empty page style (?)
    \begin{landscape}% Landscape page
        \centering % Center table
        \vfill
    \end{landscape}
    \clearpage% Flush page
}
\vfill
\clearpage




\begin{center}
{\Large Review of the Manuscript ADWR-15-232 `An Implicit Numerical Model for Multicomponent Compressible Two-Phase Flow in Porous Media' by A. Zidane and A. Firoozabadi}
\end{center}

\medskip

The manuscript describes an implicit model for compressible multiphase and multi-componet flows in porous media. Numerical simulations were performed using the new model approach with initial comparison against commercial model software under a number conditions. 

The manuscript is relatively well-written with a small number of typos and unrevised sentences (see the attached scanned document). A few sentences are confusing and disconnected with no clear objectives. However, the manuscript is comprehensive and introduces an insightful (although superficial) literature review on the main subject areas. In fact, as the manuscript mainly focuses on an implicit formulation for compressible flows, I would expect a short literature review on this subject. There is a good number of manuscripts on the fluid dynamics open literature on compressible (including shock-capturing and high-order accurate schemes), multi-component and implicit-/explicit-DG formulations. A few comments:
\begin{enumerate}[(a)] 
%
\item I understand that the authors do not want to reveal the commercial flow simulator software used for model validation, referred as CM-1/2/3. However, in order to properly appreciate the validation exercise it is necessary to know some info of this set of software, are they FEM- or FVM-based (or hybrid, CVFEM, EbFVM)? What sort of element/volume and formulation (spatial- and time-discretisation) they use?
%
\item In Eqn. 7, what does $\beta$ represent? Does it refer to the remaining phases?
%
\item The last two sentences of page 11: `The coefficient $c_{\alpha}x_{i,\alpha}$, (...) without considering higher-order spatial approximations. This is due to the fact that the pressure is a smooth function of space in the scope of this work'. What do the authors mean here? Pressure and phase velocities are coupled via the Darcy equation (Eqn. 3), thus a similar high-order approximation should be used in both velocity and pressure fields.
%
\item In the last sentence of pages 11-12, the authors stated that the upwind scheme used in the scalar transport equation can not be used in the pressure equation. Why not? So what is done in the proposed formulation?
%
\item A major issue in numerical modelling of compressible multiphase flows is the transition between incompressible (e.g., liquids) and compressible fluids due to the strong non-linear coupling between densities and pressure (i.e., speed of sound in fluids $\left(\eta^{2}=\left[\frac{\partial p}{\partial \rho}\right]\right)$. How does the formulation treat such discontinuities?
%
\item Simulations were performed in Section 4, where numerical solutions obtained from the formulation were qualitatively validated against commercial models (cross-code validation) and against an explicit DG model. The path for validation chosen by the authors seems consistent with much of the porous media open literature, however I wonder why didn't the authors try to quantitatively validate the model against simple 1D analytic solution (3-4 components) and then qualitatively validate against other models.
%
\item Last sentence of page 13, does the average number of Newton iteration refer to the convergence of the VLE sub-model (based on the PR EOS) or the actual Picard method (or non-linear iteration) of the whole set of conservative equations (pressure, saturation and compositional)? I believe the authors refers to the later, in this case, in the formulation are all fields are lumped together and solved in a single operation or in field-based stages (e.g., velocity-pressure $\rightarrow$ saturation $\rightarrow$ mass/mole fraction etc).
%
\end{enumerate}

%%%
%%% Appendix
%%%
{
  \includepdf[pages=-,fitpaper, angle=0]{./Scan/Scan_ADWR-15-232_Review.pdf}
}


%%%
%%% \lipsum % Text before
%%%
\afterpage{%
    \clearpage% Flush earlier floats (otherwise order might not be correct)
    \thispagestyle{empty}% empty page style (?)
    \begin{landscape}% Landscape page
        \centering % Center table
        \vfill
    \end{landscape}
    \clearpage% Flush page
}
\vfill
\clearpage

%%%%%%%
%%%%%%%
%%%%%%%


\begin{center}
{\Large Review of the Manuscript PNUCENE-D-15-00085 $\lq$A Comparative Study of Pool and Dry-Cask Storage System PRAs for Spent Nuclear Fuel' by {\it S. Zhang et al.}}
\end{center}

\medskip

The manuscript describes the procedure for preliminary risk assessment in nuclear fuel storage systems. The work is based on pilot studies in two nuclear power plants in China operating with distinct storage systems: pool and dry-cask. The paper is well-written with a small number of typos, a few sentences are confusing and disconnected with no clear objectives. 

In general, the manuscript is comprehensive and introduces best-practive procedures for PRA analysis for spent nuclear fuel storage.  However, my main concern is that there is no clear indication of neither scientific objectives of the paper nor novelties that the authors are bringing. In fact, the paper focuses on the main conclusions of PRA analysis, without any indication of a new methodology.  


%%%
%%% \lipsum % Text before
%%%
\afterpage{%
    \clearpage% Flush earlier floats (otherwise order might not be correct)
    \thispagestyle{empty}% empty page style (?)
    \begin{landscape}% Landscape page
        \centering % Center table
        \vfill
    \end{landscape}
    \clearpage% Flush page
}
\vfill
\clearpage

%%%%%%%
%%%%%%%
%%%%%%%

\begin{center}
  {\Large Review of the Manuscript BJCE-2015-0011 $\lq$3D Compositional Reservoir Simulation in Conjunction with Unstructured Grids' by Ara\'ujo {\it et al.}}
\end{center}

\medskip

The manuscript aims to demonstrate the functionalities of a compositional model formulation embedded in the UTCOMP flow simulator. The model is based on the element-based finite volume method (EbFVM) and benchmarks were performed with a number of element/cell types. The authors presented a brief (and limited) literature review on the current state-of-the-art of flow simulator methods and formulations and performed a set of numerical simulations. 

The manuscript is relatively well-written with a small number of typos and unrevised sentences (see the attached scanned document). A few sentences are very confusing and disconnected with no clear objectives. The method and benchmarks introduced in the manuscript are very interesting, however reading the manuscript is difficult to identify the authors' original contribution. What novelties do the manuscript bring? A few notes/questions:
\begin{enumerate}
%
\item Some terms were used before being defined, e.g., $\lq$PEBI' in the Abstract;
\item Page 4 (lines 34-47): one of the main features in the model is the use of the EbFVM instead of more traditional (for flow simulators) FEM, FDM and CVFEM. However, no specific explanation of the method is given, except $\lq$(...) The main difference between CVFEM and EbFVM techniques lies in the assumption of multiphase/multicomponent flow of the EbFVM approach in order to obtain the approximated solution (...)'. It is not clear the difference between the methods, surely neither FEM nor CVFEM are not bounded by single/multi-phase/component nature of flows. Thus I guess the author possibly meant some scalar field conservation aspect;
\item Indices in Eqn 1 and the text following it seems to be messed up; 
\item Is Eqn 2 correct? A quick dimension analysis indicates a mismatch -re $\gamma_{i}D$ term, assuming that $\gamma$ is molar weight (or molar mass, g/mol) and $D$ is reservoir depth (m);
\item Does the formulation take into account anisotropy of permeability tensors? In the tables, the diagonal components of permeability are given, thus I am assuming the synthetic geological formation is rather homogeneous, is it correct?
\item Was there any study on the convergence test performed for the formulation presented in the manuscript? 
%
\end{enumerate}
In the scanned copy,
\begin{itemize}
\item PE: Poor English. The authors should consider revise the sentence(s);
\item SC: Sentence(s) is/are very confusing and do(es) not make any/much sense. The authors should revise the sentence(s).
\end{itemize}

%%%
%%% Appendix
%%%
{
  \includepdf[pages=-,fitpaper, angle=0]{./Scan/Scan_BJCE_2015-0011.pdf}


%%%
%%% \lipsum % Text before
%%%
\afterpage{%
    \clearpage% Flush earlier floats (otherwise order might not be correct)
    \thispagestyle{empty}% empty page style (?)
    \begin{landscape}% Landscape page
        \centering % Center table
        \vfill
    \end{landscape}
    \clearpage% Flush page
}
\vfill
\clearpage

%%%%%%%
%%%%%%%
%%%%%%%


\begin{center}
{\Large Review of the Manuscript PNUCENE-D-14-00301 $\lq$CFD Study of an air-water flow inside helically coiled pipes' by {\it M. Colombo et al.}}
\end{center}

\medskip

The manuscript investigates multiphase (air-water) flow dynamics in helical pipes with a two-fluids model formulation.  Numerical simulations were performed using the FLUENT model software using a number of mesh grids and flow conditions. 

The manuscript is relatively well-written with a small number of typos and unrevised sentences (see the attached scanned document). A few sentences are confusing and disconnected with no clear objectives. However, the manuscript is comprehensive and introduces an insightful although very superficial literature review on the main subject areas. In fact, as the manuscript mainly focuses on multi-fluid dynamics in confined domains, I would expect a short literature review on this subject. There is a good number of manuscripts on the nuclear and non-nuclear open literature on both internal and external (single- and multiphase) flows and heat transfer in and around helically coiled pipes (see Y.J. Chung 2013 and 2014, Buchan 2012, all in {\it Annals}). A few comments:
\begin{enumerate}[(a)]
%
\item  The main objective of the paper (lines 27-38 of the {\it Abstract}) is to investigate flow dynamic parameters (pressure drop and void volume fraction) in helicoidal pipes. As these parameters are strongly dependent on the geometry (e.g., pipe diameter, coil pitch and curvature etc) and thermo-physical properties of the fluid (e.g., fluids' density, viscosity, velocity) I would suggest the authors to include a schematic of the helicoidal pipes. This would help understand the system the authors are simulating.
%
\item Also, as coil pitch and curvature play  crucial roles in the flow dynamics (in particular bubble formation and growth), I would expect an in-depth sensitivity analysis of these parameters.
%
\item Most of plots shown in the manuscript are functions of the {\it $\lq$quality'}, $x$. The authors have not defined this term. I am aware that flow quality is often associated with quantity of vapour phase in the two phases dome region, 
\begin{displaymath}
x_{i} = \frac{\Psi_{i} - \Psi_{\text{liq}}}{\Psi_{\text{vap}}-\Psi_{\text{liq}}}
\end{displaymath}
where $\Psi$ is any extensive thermodynamic property (e.g., specific enthalpy, entropy etc) and $i$ refers to flow conditions (i.e., pressure and temperature). Thus, in Figs. 1 and 2 (air and water experimental flow), which component does the quality refer to ? I have not read the reference paper, but I would guess that the quality refers to the amount of air in the mixture (assuming that air does not get dissolved in water), therefore the authors actually refer to the gas volume fraction. The authors should make it clear.
%
\item Last sentence of the first paragraph of Section 2. Why did the authors choose this specific experimental data set to validate their model/simulations?
%
\item In Section 3, the authors summarised the numerical options used in the simulations. As some of the readers (including myself) are not familiar with ANSYS FLUENT framework I would really suggest the authors to describe in (some) depth the methods, discretisation, schemes and solvers options used by the authors as this may explain the reason why some of the results presented seems to be slightly dissipative. 
%
\item Also, as the study focuses on pressure drop in the gas-liquid mixture (and the relationship with drag), authors should shown the equations used to represent drag and bulk density and viscosity linearisation, as they may play a significant role in the momentum conservation.
%
\item Figures 3,13 and 17 are low resolution (poor quality) and should be improved.
%
\item In Section 4, the authors performed a grid sensitivity analysis that led to the choice of mesh used in the remaining of the work. However, their procedure to choose the mesh is very random and does not make any sense. Three distinct mesh grids, three distinct inlet velocities for the gas and liquid phases and three distinct initial gas volume fraction. The resulting pressure drop {\it versus} number of cross-section cells did not shown any convergence that would indicate a grid independent solution. The authors should ensure that the mesh used in the simulations is the optimal mesh, i.e., solution is not dependent on the grid. 
%
\item Section 5: 
\begin{enumerate}[(i)]
%
\item 2$^{\text{nd}}$ paragraph: Bubble / droplet diameter are not imposed in CFD calculations (this is clear from the set of conservative equations you are solving for, where volume fraction of phases will increase/decrease depending on the flow but are also naturally constraint), however they are used as threshold in the constitutive equations to induce bubble formation, drag increase (Eqn. 5) etc.
%
\item Are the profile figures time-averaged and time-snapshots fields? I would guess the later, if this is the case, the authors should indicate the position in the helical coil and the time-stamp.
%
\item As this section investigate the influence of $d_{p}$ in the dynamics, and more specifically in the inter-phase momentum transfer, I would expect a more thorough discussion on the used drag formulation instead of just a set of simulations with different initial conditions. I believe the authors are not resolving the inter-phase momentum transfer, but rather relying on drag correlations -- so which correlation were used in the simulations?
%
\end{enumerate}
%
\item Section 6: A few CFD models treat the centrifugal term as part of gravitational term but implicitly embedded into the pressure term in the momentum equation. How is centrifugal force treated in Fluent? I would guess that this is part of the rotational reference framework in Fluent, is it correct? Please explain.
%
\item Section 7.1: The authors should explain (preferentially with the relevant equations) the wall function and the enhanced wall treatment formulations in the Fluent model. Also, was the new finer grid close to the walls regular- or exponential-based?  
%
\end{enumerate}


%%%
%%% Appendix
%%%
{
  \includepdf[pages=-,fitpaper, angle=0]{./Scan/Scan_PNUCENE_D14_00301.pdf}
}

%%%
%%% \lipsum % Text before
%%%
\afterpage{%
    \clearpage% Flush earlier floats (otherwise order might not be correct)
    \thispagestyle{empty}% empty page style (?)
    \begin{landscape}% Landscape page
        \centering % Center table
        \vfill
    \end{landscape}
    \clearpage% Flush page
}
\vfill
\clearpage

%%%%%%%
%%%%%%%
%%%%%%%



   
\end{document}
