
%\documentclass[11pts,a4paper,amsmath,amssymb,floatfix]{article}%{report}%{book}
\documentclass[12pts,a4paper,amsmath,amssymb,floatfix]{article}%{report}%{book}
\usepackage{graphicx,wrapfig,pdfpages}% Include figure files
%\usepackage{dcolumn,enumerate}% Align table columns on decimal point
\usepackage{enumerate,enumitem}% Align table columns on decimal point
\usepackage{bm,dpfloat}% bold math
\usepackage[pdftex,bookmarks,colorlinks=true,urlcolor=rltblue,citecolor=blue]{hyperref}
\usepackage{amsfonts,amsmath,amssymb,stmaryrd,indentfirst}
\usepackage{times,psfrag}
\usepackage{natbib}
\usepackage{color}
\usepackage{units}
\usepackage{rotating}
\usepackage{multirow}


\usepackage{pifont}
\usepackage{subfigure}
\usepackage{subeqnarray}
\usepackage{ifthen}

\usepackage{supertabular}
\usepackage{moreverb}
\usepackage{listings}
\usepackage{palatino}
%\usepackage{doi}
\usepackage{longtable}
\usepackage{float}
\usepackage{perpage}
\MakeSorted{figure}
%\usepackage{pdflscape}


%\usepackage{booktabs}
%\newcommand{\ra}[1]{\renewcommand{\arraystretch}{#1}}


\definecolor{rltblue}{rgb}{0,0,0.75}


%\usepackage{natbib}
\usepackage{fancyhdr} %%%%
\pagestyle{fancy}%%%%
% with this we ensure that the chapter and section
% headings are in lowercase
%%%%\renewcommand{\chaptermark}[1]{\markboth{#1}{}}
\renewcommand{\sectionmark}[1]{\markright{\thesection\ #1}}
\fancyhf{} %delete the current section for header and footer
\fancyhead[LE,RO]{\bfseries\thepage}
\fancyhead[LO]{\bfseries\rightmark}
\fancyhead[RE]{\bfseries\leftmark}
\renewcommand{\headrulewidth}{0.5pt}
% make space for the rule
\fancypagestyle{plain}{%
\fancyhead{} %get rid of the headers on plain pages
\renewcommand{\headrulewidth}{0pt} % and the line
}

\def\newblock{\hskip .11em plus .33em minus .07em}
\usepackage{color}

%\usepackage{makeidx}
%\makeindex

\setlength\textwidth      {16.cm}
\setlength\textheight     {22.6cm}
\setlength\oddsidemargin  {-0.3cm}
\setlength\evensidemargin {0.3cm}

\setlength\headheight{14.49998pt} 
\setlength\topmargin{0.0cm}
\setlength\headsep{1.cm}
\setlength\footskip{1.cm}
\setlength\parskip{0pt}
\setlength\parindent{0pt}


%%%
%%% Headers and Footers
\lhead[] {\text{\small{EX3029 -- Chemical Thermodynamics}}} 
\rhead[\text{\small{Errata 1}}]{Errata 1}
%\rfoot[] {{\text{\small{EOS + Mass Conservation using Matlab }}}}
%\chead[] {\text{\small{Session 2012/13}}} 
\lfoot[]{Dr Jeff Gomes}
\rfoot[\thepage]{\thepage}
\renewcommand{\headrulewidth}{0.8pt}


%%%
%%% space between lines
%%%
\renewcommand{\baselinestretch}{1.5}

\newenvironment{VarDescription}[1]%
  {\begin{list}{}{\renewcommand{\makelabel}[1]{\textbf{##1:}\hfil}%
    \settowidth{\labelwidth}{\textbf{#1:}}%
    \setlength{\leftmargin}{\labelwidth}\addtolength{\leftmargin}{\labelsep}}}%
  {\end{list}}

%%%%%%%%%%%%%%%%%%%%%%%%%%%%%%%%%%%%%%%%%%%
%%%%%%                              %%%%%%%
%%%%%%      NOTATION SECTION        %%%%%%%
%%%%%%                              %%%%%%%
%%%%%%%%%%%%%%%%%%%%%%%%%%%%%%%%%%%%%%%%%%%

% Text abbreviations. 
\newcommand{\lhs}{left hand side}
\newcommand{\rhs}{right hand side}
% Commands definining mathematical notation. 
\newcommand{\frc}{\displaystyle\frac}
\newcommand{\red}{\textcolor{red}}
\newcommand{\blue}{\textcolor{blue}}
\newcommand{\green}{\textcolor{green}}
\newcommand{\purple}{\textcolor{purple}}
\newcommand{\eg}{{\it e.g., }}
\newcommand{\ie}{{\it i.e., }}
\newcommand{\wrt}{{\it wrt }}
\newcommand{\Partial}[3][error]{\left(\frc{\partial #1}{\partial #2}\right)_{#3}}
\newcommand{\mfr}[3][error]{#1_{#2}^{\left(#3\right)}} 
\newcommand{\summation}[3][error]{\sum\limits_{#2}^{#3}#1}  

%%%%%%%%%%%%%%%%%%%%%%%%%%%%%%%%%%%%%%%%%%%
%%%%%%                              %%%%%%%
%%%%%% END OF THE NOTATION SECTION  %%%%%%%
%%%%%%                              %%%%%%%
%%%%%%%%%%%%%%%%%%%%%%%%%%%%%%%%%%%%%%%%%%%


% Cause numbering of subsubsections. 
%\setcounter{secnumdepth}{8}
%\setcounter{tocdepth}{8}

\setcounter{secnumdepth}{4}%
\setcounter{tocdepth}{4}%


\begin{document}

\noindent
{\Large Notes $\&$ Examples:}
\begin{itemize}
%
  \item {\bf Page 3:}  At the top of the page, $C_{p}\red{-}C_{v}=R$;
%
  \item {\bf Page 4:}  Item (iii): Derive $dQ=\frc{C_{p}}{R} PdV \red{+} \frc{C_{v}}{R} VdP$

           Again from $dU = dQ + dW$,
           \begin{displaymath}
               dQ = dU -dW = dU + PdV = C_{v}dT + PdV,
           \end{displaymath}
           Differentiating the ideal gas equation of state, $T=\frc{PV}{R}$
           \begin{displaymath}
                dT = \frc{P}{R}dV + \frc{V}{R}dP.
           \end{displaymath}
           Replacing it in the previous relation, and with $C_{p}\red{-}C_{v}=R$,
           \begin{eqnarray}
             dQ &=& C_{v}\frc{P}{R}dV + C_{v}\frc{V}{R}dP + PdV = \left(\frc{C_{v}}{R}+1\right)PdV + \frc{C_{v}}{R}VdP \nonumber \\
                &=& \red{\left(\frc{C_{p}-R}{R}+1\right)PdV + \frc{C_{v}}{R}dP} \nonumber \\
                &=&  \frc{C_{p}}{R} PdV \red{+} \frc{C_{v}}{R} VdP  \nonumber
           \end{eqnarray}
%
   \item {\bf Page 5:} Equation 11
     \begin{displaymath}
       \frc{\Delta S}{\red{R}} = \int\limits_{T_{0}}^{T_{1}} \frc{C_{p}}{R}\frc{dT}{T} - \ln{\frc{P_{1}}{P_{0}}}
     \end{displaymath}
%
   \item {\bf Page 24:} Example 2c, $\red{\Omega=0.07780}$.
%
   \item {\bf Page 28:} Equation 34d
     \begin{displaymath}
        \red{\left(\frc{\partial A}{\partial T}\right)_{V} = -S =  \left(\frc{\partial G}{\partial T}\right)_{P}}
     \end{displaymath}
%
   \item {\bf Page 43:} bottom of the page
     \begin{displaymath}
       \Partial[V]{T}{P} = -\frc{\Partial[P]{T}{V}}{\Partial[P]{V}{T}} = -\frc{\frc{R}{V-b}}{\red{-\frc{RT}{\left(V-b\right)^{2}}+\frc{2a}{V^{3}}}} 
     \end{displaymath}
%
   \item {\bf Page 19:} Close to equation 25: 
                          \begin{displaymath}
                              a = \frc{\red{0.45724} R^{2}T_{c}^{2}}{P_{c}}
                          \end{displaymath}    

   \item {\bf Page 119:}
         \begin{displaymath}
            x_{i} = \frc{n_{i,0} \red{+} \nu_{i}\varepsilon}{n_{0}+\nu\epsilon},
         \end{displaymath}   

   \item {\bf Page 121:}
         \begin{displaymath}
            y_{i} = \frc{n_{i,0} \red{+} \nu_{i}\varepsilon}{n_{0}+\nu\epsilon},
         \end{displaymath} 

   \item {\bf Page 122:}
         \begin{displaymath}
            y_{i} = \frc{n_{i,0} \red{+} \nu_{i}\varepsilon}{n_{0}+\nu\epsilon},
         \end{displaymath}     
           
%
\end{itemize}

\noindent
{\Large Mock-Exam:}
\begin{itemize}
% 
  \item Question 1(b): \red{$a=$9.126$\times$10$^{-3}$ m$^{6}$.bar.mol$^{-2}$}. 
  \item Solution of Question 1(b): Integration limits are from 10$^{-3}$ to 0.04 \red{m$^{3}$.mol$^{-1}$} and the final solution is \red{$W= \Delta U - Q = 889.19\text{ kJ.mol}^{-1}$}.
  \item Question 4: 'with [P] = kPa'.
%
\end{itemize}


\end{document}
