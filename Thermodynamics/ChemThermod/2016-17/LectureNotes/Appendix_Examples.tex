
\chapter{A Few Examples}

  \begin{list}{\bf Example \arabic{qcounter}:~}{\usecounter{qcounter}}
%
     %%% EXAMPLE 1:
     \item\label{example1} Using the cyclic rule (Appendix~\ref{Appendix_Calculus:Properties}) and the definitions,
    \begin{displaymath}
        \alpha = \frc{1}{V}\left(\frc{\partial V}{\partial T}\right)_{P} \hspace{1cm}\text{ and }\hspace{1cm} \beta = -\frc{1}{V}\left(\frc{\partial V}{\partial P}\right)_{T},
    \end{displaymath}
    \noindent show that 
    \begin{displaymath}
      \left(\frc{\partial P}{\partial T}\right)_{V} = \frc{\alpha}{\beta}.
    \end{displaymath}
%%
\medskip
     {\bf Solution:} From the cyclic rule,
       \begin{displaymath}
          \left(\frc{\partial P}{\partial T}\right)_{V}\left(\frc{\partial V}{\partial P}\right)_{T}\left(\frc{\partial T}{\partial V}\right)_{T} = -1.
       \end{displaymath}
     Thus,
       \begin{displaymath}
          \left(\frc{\partial P}{\partial T}\right)_{V} = \frc{-1}{\left(\frc{\partial V}{\partial P}\right)_{T}\left(\frc{\partial T}{\partial V}\right)_{T}} = \frc{-\left(\frc{\partial V}{\partial T}\right)_{P}}{\left(\frc{\partial V}{\partial P}\right)_{T}} = \frc{-V\alpha}{-V\beta} = \frc{\alpha}{\beta}
       \end{displaymath}
      
%
     %%% EXAMPLE 2:
     \item\label{example2} For a van der Waals gas, the pressure $P$ and the internal energy $U$ can be expressed as functions of the number of mols ($n$), total volume ($V$) and temperature ($T$),
       \begin{displaymath}
         P = \frc{n R T}{V-nb} - \frc{n^{2}a}{V^{2}} \hspace{1cm}\text{ and }\hspace{1cm} U = \frc{3}{2}n R T - \frc{n^{2}a}{V},
       \end{displaymath}
       respectively, where $a$ and $b$ are constants. Use these equations and the chain rule to derive an equation for $\left(\frc{\partial U}{\partial P}\right)_{n,T}$ in terms of $n$, $V$ and $T$.

%%
\medskip
{\bf Solution:}
   \begin{eqnarray}
      \Partial[U]{P}{n,T} &=& \Partial[U]{V}{n,T}\Partial[V]{P}{n,T} = \frc{\Partial[U]{V}{n,T}}{\Partial[P]{V}{n,T}} \nonumber \\
                          &=& \frc{\frc{n^{2}a}{V^{2}}}{\frc{2n^{2}a}{V^{3}}-\frc{n R T}{\left(V-nb\right)^{2}}} = \frc{n a}{\frc{2 n a}{V}-\frc{R T V^{2}}{\left(V-nb\right)^{2}}}\nonumber
   \end{eqnarray}
      
%
     %%% EXAMPLE 3:
     \item\label{example3} The heat capacity at constant volume is defined as $C_{v}\equiv \Partial[U]{T}{V}$. Show that
       \begin{displaymath}
          \Partial[U]{T}{P} = C_{v} + \alpha V\Partial[U]{V}{T},
       \end{displaymath}
       with $\alpha=\frc{1}{V}\Partial[V]{T}{P}$.

%
\medskip
       {\bf Solution:}
          \begin{displaymath}
            \Partial[U]{T}{P} = \Partial[U]{T}{V} + \Partial[U]{V}{T}\Partial[V]{T}{P},
          \end{displaymath}
          however $\Partial[U]{T}{V}=C_{v}$ and $\Partial[V]{T}{P}=V\alpha$. Thus,
          \begin{displaymath}
             \Partial[U]{T}{P} = C_{v} + \alpha V\Partial[U]{V}{T}.
          \end{displaymath}

%
     %%% EXAMPLE 4:
     \item\label{example4} h
%
\end{list}
