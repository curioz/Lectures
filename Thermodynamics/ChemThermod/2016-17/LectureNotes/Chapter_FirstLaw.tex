
%%%
%%% CHAPTER
%%%
\chapter{First Law of Thermodynamics}\label{Chapter:FirstLaw}

   \begin{LearningObjectivesBlock}{Learning Objectives}
      Upon completion of this chapter, you will be able to
        \begin{enumerate}
           \item Demonstrate understanding of key concepts of energy and the first law of thermodynamics;
           \item Apply the first law of thermodynamics to assess of heat transfer and power cycles;
           \item Conduct energy analysis of thermodynamic systems;
           \item Employ energy and mass balances into thermodynamic systems to assess efficiency, and correctly observe sign conventions for work and heat transfer.
        \end{enumerate}
\medskip
     Recommended reading: Chapters 2 of \citet{Atkins_Book,SmithVanNess_Book,Moran_Book} or 3 of \citet{Borgnakke_Book}.
   \end{LearningObjectivesBlock}

%%%%%%%%%%%%%%%%%%%%%%%%%%%%%%%%%%%%%%%%%%%%%%%%%%%%%%%%%%%%%%%%%
\begin{comment}
   \begin{LearningObjectivesBlock}{Learning Objectives}
      Upon completion of this chapter, you will be able to
        \begin{enumerate}
           \item {\bf Knowledge:} Define, Name, Select, State 
           \item {\bf Comprehension:} Describe, Identify, Discuss
           \item {\bf Application:} Apply, Demonstrate, Employ, Sketch
           \item {\bf Analysis:} Analyse, Compare, Calculate, Solve
           \item {\bf Synthesis:} Determine, Formulate
           \item {\bf Evaluation:} Assess, Check, Estimate, Compare, Measure, Monitor
        \end{enumerate}
\medskip
     Recommended reading: Chapters 2 of \citet{Atkins_Book,SmithVanNess_Book,Moran_Book} or 3 of \citet{Borgnakke_Book}.
   \end{LearningObjectivesBlock}
\end{comment}

   
%%%
%%% SECTION
%%%
     \section{Introduction}\label{Chapter:FirstLaw:Section:Intro}\index{Work}\index{Heat}\index{Energy}
     In Section~\ref{Chapter:Introduction:Section:ThermodAnalysis}, the main elements in the thermodynamic analysis were introduced, namely {\bf open, closed and isolated systems}, {\bf surroundings} and {\bf boundaries}. The concept of {\bf energy}, {\bf work} and {\bf heat}, pivotal entities in the study of thermodynamics systems, were also defined as,
     \begin{description}
        \item[Work] is motion against an opposing force (Eqn.~\ref{Chpt01_Work1});
        \item[Energy] of a system is its capacity to produce work, and; 
        \item[Heat] is the transfer of energy across the boundary caused by a temperature gradient at the boundary \citep{Devoe_Book}.
     \end{description}
     These definitions are based on observations of systems in a macro-scale, and are critical for mass and energy balances necessary for this chapter. ADD MORE TEXT HERE !!

%%%
%%% SECTION
%%%
     \section{The Internal Energy}\label{Chapter:FirstLaw:Section:ThermalEnergy}\index{Internal Energy}\index{Energy!Internal|see{Internal Energy}}
     \begin{subequations}
        A system, with a prescribed amount of mass, contains energy in the form of {\bf internal energy} ($U$, inherent in the internal structure), kinetic energy (linked to the motion) and potential energy (associated with external forces acting upon the mass). The total energy, $E$, associated of the system can then be expressed as 
        \begin{displaymath}
            E = \text{Internal} + \text{Kinetic} + \text{Potential} = U + E_{\text{K}} + E_{\text{P}},
        \end{displaymath}
        and the specific energy, $e$, becomes
        \begin{equation}
            e = \frc{E}{m} = u + e_{\text{K}} + e_{\text{P}} = u + \frc{1}{2}v^{2} + gz,\label{Chapter:FirstLaw:Eqn:TotalEnergy}
        \end{equation}
        where the kinetic energy\footnote{Kinetic energy has three components: vibrational (due to the energy associated with vibration of the body), rotational (associated with the rotation motion) and translational (associated with the motion from one spatial coordinate to another).} is assumed to be due to the translational motion (thus vibrational and rotational motion are neglected) and the potential energy to be due to the constant gravitational force. In Eqn.~\ref{Chapter:FirstLaw:Eqn:TotalEnergy}, $u$, $e_{\text{K}}$ and $e_{\text{P}}$ are specific internal, kinetic and potential energies, respectively. Kinetic and potential energy are associated with the physical state and location (spatial coordinates) of the system, and are commonly named {\it mechanical energy}\index{Energy!Mechanical}.  The internal energy is a characteristic of the thermodynamic state of the mass and is often labelled as {\it thermal energy}\index{Energy!Thermal}.
      
       \begin{shaded}
          The internal energy is a {\it state function}, as its value depends only on the current state of the system and is independent of processes undertook by the system, \ie it is a function of the properties that determine the current state of the system.
       \end{shaded}


     \end{subequations}
