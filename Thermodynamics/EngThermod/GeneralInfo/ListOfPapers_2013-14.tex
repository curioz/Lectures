

%\documentclass[11pts,a4paper,amsmath,amssymb,floatfix]{article}%{report}%{book}
%\documentclass[12pts,a4paper,amsmath,amssymb,floatfix]{report}%{article}%{report}%{book}
\documentclass[12pts,a4paper]{report}%{article}%{report}%{book}
%\usepackage{dcolumn,enumerate}% Align table columns on decimal point
\usepackage{enumerate,enumitem}% Align table columns on decimal point
\usepackage{bm,dpfloat}% bold math
\usepackage[pdftex,bookmarks,colorlinks=true,urlcolor=rltblue,citecolor=blue]{hyperref}
\usepackage{amsfonts,amsmath,amssymb,stmaryrd,indentfirst}
\usepackage{times,psfrag}
\usepackage{color}
\usepackage{units}
\usepackage{listings}
\usepackage{palatino}
%\usepackage{doi}
\usepackage{float}
\usepackage{perpage}
\MakeSorted{figure}
%\usepackage{pdflscape}


\definecolor{rltblue}{rgb}{0,0,0.75}


%\usepackage{natbib}
\usepackage{fancyhdr} %%%%
\pagestyle{fancy}%%%%
% with this we ensure that the chapter and section
% headings are in lowercase
%%%%\renewcommand{\chaptermark}[1]{\markboth{#1}{}}
\renewcommand{\sectionmark}[1]{\markright{\thesection\ #1}}
\fancyhf{} %delete the current section for header and footer
\fancyhead[LE,RO]{\bfseries\thepage}
\fancyhead[LO]{\bfseries\rightmark}
\fancyhead[RE]{\bfseries\leftmark}
\renewcommand{\headrulewidth}{0.5pt}
% make space for the rule
\fancypagestyle{plain}{%
\fancyhead{} %get rid of the headers on plain pages
\renewcommand{\headrulewidth}{0pt} % and the line
}

\def\newblock{\hskip .11em plus .33em minus .07em}
\usepackage{color}

%\usepackage{makeidx}
%\makeindex

\setlength\textwidth      {16.cm}
\setlength\textheight     {22.6cm}
\setlength\oddsidemargin  {-0.3cm}
\setlength\evensidemargin {0.3cm}

\setlength\headheight{14.49998pt} 
\setlength\topmargin{0.0cm}
\setlength\headsep{1.cm}
\setlength\footskip{1.cm}
\setlength\parskip{0pt}
\setlength\parindent{0pt}


%%%
%%% Headers and Footers
\lhead[] {\text{\small{EG3521 -- Engineering Thermodynamics}}} 
\rhead[] {{\text{\small{Continuous Assessment (2013/14)}}}}
%\chead[] {\text{\small{Session 2012/13}}} 
\lfoot[]{Continuous Assessment}
%\cfoot[\thepage]{\thepage}
\rfoot[\text{\small{\thepage}}]{\thepage}
\renewcommand{\headrulewidth}{0.8pt}


%%%
%%% space between lines
%%%
\renewcommand{\baselinestretch}{1.5}

\newenvironment{VarDescription}[1]%
  {\begin{list}{}{\renewcommand{\makelabel}[1]{\textbf{##1:}\hfil}%
    \settowidth{\labelwidth}{\textbf{#1:}}%
    \setlength{\leftmargin}{\labelwidth}\addtolength{\leftmargin}{\labelsep}}}%
  {\end{list}}

%%%%%%%%%%%%%%%%%%%%%%%%%%%%%%%%%%%%%%%%%%%
%%%%%%                              %%%%%%%
%%%%%%      NOTATION SECTION        %%%%%%%
%%%%%%                              %%%%%%%
%%%%%%%%%%%%%%%%%%%%%%%%%%%%%%%%%%%%%%%%%%%

% Text abbreviations.
\newcommand{\ie}{{\em{i.e., }}}
\newcommand{\eg}{{\em{e.g., }}}
\newcommand{\cf}{{\em{cf., }}}
\newcommand{\wrt}{with respect to}
\newcommand{\lhs}{left hand side}
\newcommand{\rhs}{right hand side}
% Commands definining mathematical notation.

% This is for quantities which are physically vectors.
\renewcommand{\vec}[1]{{\mbox{\boldmath$#1$}}}
% Physical rank 2 tensors
\newcommand{\tensor}[1]{\overline{\overline{#1}}}
% This is for vectors formed of the value of a quantity at each node.
\newcommand{\dvec}[1]{\underline{#1}}
% This is for matrices in the discrete system.
\newcommand{\mat}[1]{\mathrm{#1}}


\DeclareMathOperator{\sgn}{sgn}
\newtheorem{thm}{Theorem}[section]
\newtheorem{lemma}[thm]{Lemma}

%\newcommand\qed{\hfill\mbox{$\Box$}}
\newcommand{\re}{{\mathrm{I}\hspace{-0.2em}\mathrm{R}}}
\newcommand{\inner}[2]{\langle#1,#2\rangle}
\renewcommand\leq{\leqslant}
\renewcommand\geq{\geqslant}
\renewcommand\le{\leqslant}
\renewcommand\ge{\geqslant}
\renewcommand\epsilon{\varepsilon}
\newcommand\eps{\varepsilon}
\renewcommand\phi{\varphi}
\newcommand{\bmF}{\vec{F}}
\newcommand{\bmphi}{\vec{\phi}}
\newcommand{\bmn}{\vec{n}}
\newcommand{\bmns}{{\textrm{\scriptsize{\boldmath $n$}}}}
\newcommand{\bmi}{\vec{i}}
\newcommand{\bmj}{\vec{j}}
\newcommand{\bmk}{\vec{k}}
\newcommand{\bmx}{\vec{x}}
\newcommand{\bmu}{\vec{u}}
\newcommand{\bmv}{\vec{v}}
\newcommand{\bmr}{\vec{r}}
\newcommand{\bma}{\vec{a}}
\newcommand{\bmg}{\vec{g}}
\newcommand{\bmU}{\vec{U}}
\newcommand{\bmI}{\vec{I}}
\newcommand{\bmq}{\vec{q}}
\newcommand{\bmT}{\vec{T}}
\newcommand{\bmM}{\vec{M}}
\newcommand{\bmtau}{\vec{\tau}}
\newcommand{\bmOmega}{\vec{\Omega}}
\newcommand{\pp}{\partial}
\newcommand{\kaptens}{\tensor{\kappa}}
\newcommand{\tautens}{\tensor{\tau}}
\newcommand{\sigtens}{\tensor{\sigma}}
\newcommand{\etens}{\tensor{\dot\epsilon}}
\newcommand{\ktens}{\tensor{k}}
\newcommand{\half}{{\textstyle \frac{1}{2}}}
\newcommand{\tote}{E}
\newcommand{\inte}{e}
\newcommand{\strt}{\dot\epsilon}
\newcommand{\modu}{|\bmu|}
% Derivatives
\renewcommand{\d}{\mathrm{d}}
\newcommand{\D}{\mathrm{D}}
\newcommand{\ddx}[2][x]{\frac{\d#2}{\d#1}}
\newcommand{\ddxx}[2][x]{\frac{\d^2#2}{\d#1^2}}
\newcommand{\ddt}[2][t]{\frac{\d#2}{\d#1}}
\newcommand{\ddtt}[2][t]{\frac{\d^2#2}{\d#1^2}}
\newcommand{\ppx}[2][x]{\frac{\partial#2}{\partial#1}}
\newcommand{\ppxx}[2][x]{\frac{\partial^2#2}{\partial#1^2}}
\newcommand{\ppt}[2][t]{\frac{\partial#2}{\partial#1}}
\newcommand{\pptt}[2][t]{\frac{\partial^2#2}{\partial#1^2}}
\newcommand{\DDx}[2][x]{\frac{\D#2}{\D#1}}
\newcommand{\DDxx}[2][x]{\frac{\D^2#2}{\D#1^2}}
\newcommand{\DDt}[2][t]{\frac{\D#2}{\D#1}}
\newcommand{\DDtt}[2][t]{\frac{\D^2#2}{\D#1^2}}
% Norms
\newcommand{\Ltwo}{\ensuremath{L_2} }
% Basis functions
\newcommand{\Qone}{\ensuremath{Q_1} }
\newcommand{\Qtwo}{\ensuremath{Q_2} }
\newcommand{\Qthree}{\ensuremath{Q_3} }
\newcommand{\QN}{\ensuremath{Q_N} }
\newcommand{\Pzero}{\ensuremath{P_0} }
\newcommand{\Pone}{\ensuremath{P_1} }
\newcommand{\Ptwo}{\ensuremath{P_2} }
\newcommand{\Pthree}{\ensuremath{P_3} }
\newcommand{\PN}{\ensuremath{P_N} }
\newcommand{\Poo}{\ensuremath{P_1P_1} }
\newcommand{\PoDGPt}{\ensuremath{P_{-1}P_2} }

\newcommand{\metric}{\tensor{M}}
\newcommand{\configureflag}[1]{\texttt{#1}}

% Units
\newcommand{\m}[1][]{\unit[#1]{m}}
\newcommand{\km}[1][]{\unit[#1]{km}}
\newcommand{\s}[1][]{\unit[#1]{s}}
\newcommand{\invs}[1][]{\unit[#1]{s}\ensuremath{^{-1}}}
\newcommand{\ms}[1][]{\unit[#1]{m\ensuremath{\,}s\ensuremath{^{-1}}}}
\newcommand{\mss}[1][]{\unit[#1]{m\ensuremath{\,}s\ensuremath{^{-2}}}}
\newcommand{\K}[1][]{\unit[#1]{K}}
\newcommand{\PSU}[1][]{\unit[#1]{PSU}}
\newcommand{\Pa}[1][]{\unit[#1]{Pa}}
\newcommand{\kg}[1][]{\unit[#1]{kg}}
\newcommand{\rads}[1][]{\unit[#1]{rad\ensuremath{\,}s\ensuremath{^{-1}}}}
\newcommand{\kgmm}[1][]{\unit[#1]{kg\ensuremath{\,}m\ensuremath{^{-2}}}}
\newcommand{\kgmmm}[1][]{\unit[#1]{kg\ensuremath{\,}m\ensuremath{^{-3}}}}
\newcommand{\Nmm}[1][]{\unit[#1]{N\ensuremath{\,}m\ensuremath{^{-2}}}}

% Dimensionless numbers
\newcommand{\dimensionless}[1]{\mathrm{#1}}
\renewcommand{\Re}{\dimensionless{Re}}
\newcommand{\Ro}{\dimensionless{Ro}}
\newcommand{\Fr}{\dimensionless{Fr}}
\newcommand{\Bu}{\dimensionless{Bu}}
\newcommand{\Ri}{\dimensionless{Ri}}
\renewcommand{\Pr}{\dimensionless{Pr}}
\newcommand{\Pe}{\dimensionless{Pe}}
\newcommand{\Ek}{\dimensionless{Ek}}
\newcommand{\Gr}{\dimensionless{Gr}}
\newcommand{\Ra}{\dimensionless{Ra}}
\newcommand{\Sh}{\dimensionless{Sh}}
\newcommand{\Sc}{\dimensionless{Sc}}


% Journals
\newcommand{\IJHMT}{{\it International Journal of Heat and Mass Transfer}}
\newcommand{\NED}{{\it Nuclear Engineering and Design}}
\newcommand{\ICHMT}{{\it International Communications in Heat and Mass Transfer}}
\newcommand{\NET}{{\it Nuclear Engineering and Technology}}
\newcommand{\HT}{{\it Heat Transfer}}   
\newcommand{\IJHT}{{\it International Journal for Heat Transfer}}

\newcommand{\frc}{\displaystyle\frac}

\newlist{ExList}{enumerate}{1}
\setlist[ExList,1]{label={\bf Example 1.} {\bf \arabic*}}

\newlist{ProbList}{enumerate}{1}
\setlist[ProbList,1]{label={\bf Problem 1.} {\bf \arabic*}}

%%%%%%%%%%%%%%%%%%%%%%%%%%%%%%%%%%%%%%%%%%%
%%%%%%                              %%%%%%%
%%%%%% END OF THE NOTATION SECTION  %%%%%%%
%%%%%%                              %%%%%%%
%%%%%%%%%%%%%%%%%%%%%%%%%%%%%%%%%%%%%%%%%%%


% Cause numbering of subsubsections. 
%\setcounter{secnumdepth}{8}
%\setcounter{tocdepth}{8}

\setcounter{secnumdepth}{4}%
\setcounter{tocdepth}{4}%




\begin{document}

\begin{center}
{\Large Continuous Assessment -- Group Work}
\end{center}

\begin{enumerate}
%
\item {\bf 16 Groups of maximum 6 students};
%
\item Choose one of the Manuscripts in the list below and assign yourselves in the {\it MyAberdeen}'s Groups -- Deadline: February 27$^{th}$.
%
\item Assessment: Oral Presentation and Discussion (May 6-9$^{th}$):
  \begin{itemize}
   %\item \textcolor{red}{Everyone must attend the {\bf ALL} presentations and participate in the discussions}
   \item 10 minutes presentation following 5 minutes of QA and discussions
   \item Groups 1--4: May 6; 
   \item Groups 5--8: May 8 (morning);
   \item Groups 9--12: May 8 (afternoon);
   \item Groups 13--16: May 9;
  \end{itemize}
 \item Feedback will be given 1 week after the last presentation (before May 16);
 \item The presentation should focus on the thermodynamics aspects of the manuscript and it must cover:
 \begin{itemize}
   \item Summary of the manuscript \textcolor{red}{(not copy and paste)};
   \item Relevance (i.e., main motivation for the work, applications etc);
   \item How the manuscript relates with the Thermodynamics course;
   %\item How the main subject of the manuscript can be communicate to a broad audience;
   \item How the subject can be effectively integrated into a thermodynamic lecture;
\end{itemize}
\item {\bf The slides presentation should be submitted (as PDF files) by email \textcolor{red}{before May 2$^{nd}$ at 5pm}.}  The $\lq$Subject' of the email (\href{mailto:jefferson.gomes@abdn.ac.uk}{jefferson.gomes@abdn.ac.uk}) must be $\lq$EG3521 Presentation - Group XX'.
\end{enumerate}
%


\clearpage

\begin{center}
{\Large List of Groups / Manuscripts}
\end{center}


\begin{enumerate}[label=\bfseries Group \arabic*:]
%
\item A.A. Hasan, D.Y. Goswami and S. Vijayaraghavan (2002) \href{http://dx.doi.org/10.1016/S0038-092X(02)00113-5}{$\lq$First and Second Law Analysis of a New Power and Refrigeration Thermodynamic Cycle using a Solar Heat Source'}, {\it Solar Energy} 73:385-393.
%
\item D. Zhao, G. Ding and Z. Wu (2009) \href{http://dx.doi.org/10.1016/j.ijrefrig.2009.05.005}{$\lq$Extension of the implicit curve-fitting method for fast calculation of thermodynamic properties of refrigerants in supercritical region'}, {\it International Journal of Refrigeration}, 32:1615-1625.
%
\item C. Coquelet and D. Richon (2009) \href{http://dx.doi.org/10.1016/j.ijrefrig.2009.03.013}{$\lq$Experimental determination of phase diagram and modeling: Application to refrigerant mixtures'}, {\it International Journal of Refrigeration}, 32:1604-1614.

\item M.E. Duprez, E. Dumont and M. Fr\`ere (2007) \href{http://dx.doi.org/10.1016/j.ijrefrig.2006.11.014}{$\lq$Modelling of Reciprocating and Scroll Compressors'}, {\it International Journal of Refrigeration}, 30:873-886.
%
\item J. Tarlecki, N. Lior and N. Zhang (2007) \href{http://dx.doi.org/10.1016/j.enconman.2007.06.039}{$\lq$Analysis of Thermal Cycles and Working Fluids for Power Generation in Space'}, {\it Energy Conversion and Management}, 48:2864-2878.
%
\item R.N. Christensen and M. Santoso (1990) \href{http://dx.doi.org/10.1016/0890-4332(90)90169-K}{$\lq$An Evaluation of a Rankine Cycle Driven Heat Pump'}, {\it Heat Recovery $\&$ CHP}, 10:161-175.
%
\item R.F. Garcia (2012) \href{http://dx.doi.org/10.1016/j.applthermaleng.2012.02.039}{$\lq$Efficiency Enhancement of Combined Cycles by Suitable Working Fluids and Operating Conditions'}, {\it Applied Thermal Engineering}, 42:25-33.
%
\item M.G. Soufi, T. Fujii, K. Sugimoto and H. Asano (2004) \href{http://inderscience.metapress.com/content/vvmpkbwp3x0fk08c/?genre=article&issn=1742-8297&volume=1&issue=1&spage=29}{$\lq$A New Rankine Cycle for Hydrogen-Fired Power Generation Plants and Its Exergetic Efficiency'}, {\it International Journal of Exergy} 1:29-46.
%
\item H. Chen, D.Y. Goswami and W.K. Stefanakos (2010) \href{http://dx.doi.org/10.1016/j.rser.2010.07.006}{$\lq$A Review of Thermodynamic Cycles and Working Fluids for the Conversion of Low-Grade Heat'}, {\it Renewable and Sustainable Energy Reviews}, 14:3059-3067.
%
\item G. Tsiklauri, R. Talbert, B. Schmitt, G. Filippov, R. Bogoyavlensky and E. Grishanin (2005) \href{http://dx.doi.org/10.1016/j.nucengdes.2004.11.016}{$\lq$Supercritical Steam Cycle for Nuclear Power Plant'}, {\it Nuclear Engineering and Design}, 235:1651-1664.
%
\item Y. Le Moullec (2013) \href{http://dx.doi.org/10.1016/j.energy.2012.10.022}{$\lq$Conceptual Study of a High Eficiency Coal-Fired Power Plant with CO$_{2}$ capture using Supercritical CO$_{2}$ Brayton Cycle'}, {\it Energy}, 49:32-46.
%
\item L. Lizon-A-Lugrin, A. Teyssedou and I. Pioro (2012) \href{http://dx.doi.org/10.1016/j.nucengdes.2011.07.024}{$\lq$Appropriate Thermodynamic Cycles to be used in Future Pressure-Channel Supercritical Water-Cooled Nuclear Power Plants'}, {\it Nuclear Engineering and Design}, 246:2-11.
%
\item A.R.D. Thorley and C.H. Tiley (1987) \href{http://dx.doi.org/10.1016/0142-727X(87)90044-0}{$\lq$Unsteady and transient flow of compressible fluids in pipelines—a review of theoretical and some experimental studiess'}, {\it Int. J. Heat Fluid Flow} 8(1):3–15.
%
\item P. Spittle (2003) \href{http://dx.doi.org/10.1088/0031-9120/38/6/002}{$\lq$Gas Turbine Technology'}, {\it Physics Education}, 38:504-511.
%
\item J. Xu, Z. Wu, S. Wang, B. Qi, K. Chen, X. Li and X. Zhao (2013) \href{http://onlinelibrary.wiley.com/doi/10.1002/cjce.20689/abstract}{$\lq$Prediction of temperature, pressure, density, velocity distribution in H-T-H-P gas wells'}, {\it Canadian J. Chem. Eng.} 91(1):111-121.
%
\item D. Hullender, R. Woods and Y.-W. Huang (2010) \href{http://dx.doi.org/10.1115/1.4000742}{$\lq$Single Phase Compressible Steady Flow in Pipes'}, {\it J. Fluid. Eng.} 132(1):014502.
%
%\item 
%\begin{enumerate}
%  \item C. Jones and G. Wake (2012) $\lq$New Zealand at a Carbon Sequestration Advantage! -- Part 1', {\it Chemistry Education in New Zealand}, February, 5-6; 
%
%\item C. Jones and G. Wake (2012) $\lq$New Zealand at a Carbon Sequestration Advantage! -- Part 2', {\it Chemistry Education in New Zealand}, February, 7. 
%
%\end{enumerate}
\end{enumerate}

%\section{Backup Manuscripts}
%\begin{enumerate}
%
%\item R. Nieto, C. Gonz\'alez, A. Jim\'enez, I. L\'opez and J. Rodr\'{\i}guez (2011) \href{http://dx.doi.org/10.1021/ed100798p}{$\lq$A Missing Deduction of the Clausius Equality and Inequality'}, {\it Journal of Chemical Education}, 88:597-601.
%
%\item N.C. Craig and E.A. Gislason (2002) \href{http://pubs.acs.org/doi/abs/10.1021/ed079p193}{$\lq$First Law of Thermodynamics; Irreversible and Reversible Processes'}, {\it Journal of Chemical Education}, 79:193-200.
%
%\item C. Salter (2000) \href{http://pubs.acs.org/doi/abs/10.1021/ed077p1027}{$\lq$A Simple Approach to Heat Engine Efficiency'}, {\it Journal of Chemical Education}, 77:1027-1030.
%
%\end{enumerate}



%{
%  \includepdf[pages=-,fitpaper]{EG3521_OralPresentationAssessmentSheet.pdf}
%}

\end{document}
