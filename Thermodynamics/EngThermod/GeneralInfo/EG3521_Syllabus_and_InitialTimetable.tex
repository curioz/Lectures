\documentclass[11pt,oneside,a4paper]{article}

%\usepackage{amsfonts,amsmath,amssymb,stmaryrd,indentfirst}
%\usepackage{psfig,graphicx,times,psfrag}
\usepackage{amsfonts,amsmath,amssymb,indentfirst}
%\usepackage{graphicx,times,psfrag}
\usepackage{graphicx,times}
\usepackage{enumerate}% Align table columns on decimal point
%\usepackage{bm,dpfloat}% bold math
\usepackage[pdftex,bookmarks,colorlinks=true,urlcolor=blue,citecolor=blue]{hyperref}
\usepackage{multirow,color,longtable}

%\usepackage{natbib}
%\usepackage{fancyhdr} %%%%
%\pagestyle{fancy}%%%%
\def\newblock{\hskip .11em plus .33em minus .07em}

\setlength\textwidth      {18.5cm}
\setlength\textheight     {26.2cm}
\setlength\oddsidemargin  {-1.1cm}
\setlength\evensidemargin {0.6cm}

\setlength\headheight{-.1cm}
%%\setlength\headheight{14.49998pt}
\setlength\topmargin{-2.2cm}
\setlength\headsep{1.cm}
\setlength\footskip{.9cm}
\setlength\parskip{0pt}

%%%
%%% space between lines
%%%
\renewcommand{\baselinestretch}{0.9}
\begin{document}

\setcounter{page}{1}

\begin{center}
{\bf {\Large Engineering Thermodynamics (EG3521)}}\\
{\large Drs Jeff Gomes (Course Coordinator) and Peter Hicks}
\end{center}


\section{ Syllabus }

\begin{enumerate}[{\bf Module 1:}]
%
\item Introduction and Principles (2 lectures, Chps: 2-4 of \cite{smith_2001}, 2, 3 and 5 of \cite{borgnakke_2013}):
\begin{enumerate}[(a)]
\item First and second laws;
\item Ideal gas processes;
\item Steam tables; 
\end{enumerate}

\item The production of power from heat (7 lectures, Chps: 8 of \cite{smith_2001}, 9 and 10 of \cite{cengel_2010}, 12 and 13 of \cite{rajput_2007}): 
\begin{enumerate}[(a)]
\item Revision of the Carnot-engine cycle and the basic steam power plant; the Rankine cycle; 
\item Modifications of the Rankine cycle to increase efficiency including superheat/re-heat $\&$ feed water heating; 
\item Internal combustion engines including the Otto Engine, the Diesel engine and the gas-turbine engine (Brayton cycle); 
\end{enumerate}
%
\item Refrigeration $\&$ Liquefaction (5 lectures, Chps: 9 of \cite{smith_2001}, 11 of \cite{cengel_2010}, 14 of \cite{rajput_2007}):
\begin{enumerate}[(a)]
\item The Carnot refrigerator; 
\item The vapour-compression cycle; 
\item Refrigerant choice; 
\item The Linde liquefaction process; 
\item The Claude liquefaction process;
\end{enumerate}
%
\item Applications of thermodynamics to flow processes (6 lectures, Chps: 7 of \cite{smith_2001} and 5 and 17 of \cite{cengel_2010}, 16 of \cite{rajput_2007}, 6 and 9 of \cite{powers_2012}): 
\begin{enumerate}[(a)]
\item Duct flow of compressible fluids in pipes, nozzles and throttling devices; 
\item Turbines; 
\item Compression processes including compressors, pumps and ejectors ;
\end{enumerate}
%
\item Psychrometry (3 lectures, Chps: 10 of \cite{muller_2009}, 14 of \cite{cengel_2010}, 10 of \cite{rajput_2007}): 
\begin{enumerate}[(a)]
\item Basic definitions; 
\item Wet bulb temperature; 
\item Adiabatic saturation temperature; 
\item Humidity data for the air-water system (humidity-temperature chart $\&$ humidity-enthalpy diagram); 
\item Mixing of humid streams; 
\item Humidification $\&$ dehumidification.
\end{enumerate}

\end{enumerate}

\section{Timetable}


\begin{center}
\begin{tabular}{||c||c|c|c|c|c||}
\hline\hline
\multicolumn{6}{||c||}{Engineering Thermodynamics (EG3521)} \\
\hline\hline
\multirow{3}{*}{\color{red}{Week 30}} & Feb 03   & 12.00-13.00 & Intro Lect 1.1 & JG  & KCG8 \\
                                      & Feb 04   & 09.00-10.00 & Lect 1.2  & JG  & FN2  \\
                                      & Feb 06   & 12.00-13.00 & Lect 1.3 + \textcolor{red}{Tutorial 1} & JG  & MT4  \\
\hline
\multirow{3}{*}{\color{red}{Week 31}} & Feb 10   & 12.00-13.00 & Lect  2.1  & JG  & KCG8 \\
                                      & Feb 11   & 09.00-10.00 & Lect  2.2  & JG  & FN2 \\
                                      & Feb 13   & 12.00-13.00 &  --        & --    & \textcolor{red}{No Lecture} \\
\hline
\multirow{3}{*}{\color{red}{Week 32}} & Feb 17   & 12.00-13.00 & --         & --     & \textcolor{red}{No Lecture} \\
                                      & Feb 18   & 09.00-10.00 & Lect  2.3  & JG  & FN2  \\
                                      & Feb 20   & 12.00-13.00 & Lect  2.4  & JG  & MT4  \\
                                      & Feb 20   & 09.00-10.00 & \textcolor{blue}{Cont. Assessment 1} & All & G3 St. Mary's \\
                                      & Feb 21   & 16.00-17.00 & \textcolor{red}{Tutorial 2} & JG  & 105 St Mary's (optional)\\
\hline
\multirow{3}{*}{\color{red}{Week 33}} & Feb 24   & 12.00-13.00 & Lect  2.5  & JG  & KCG8 \\
                                      & Feb 25   & 09.00-10.00 & Lect  2.6  & JG  & FN2  \\
                                      & Feb 27   & 12.00-13.00 & \textcolor{red}{Tutorial 3} & JG  & MT4  \\
                                      & Feb 27   & 09.00-10.00 & \textcolor{blue}{Cont. Assessment 2} & All  &  G3 St. Mary's \\
                                      & Feb 28   & 16.00-17.00 & \textcolor{red}{Tutorial 4} & JG  & 105 St Mary's (optional)\\
\hline
\multirow{3}{*}{\color{red}{Week 34}} & Mar 03   & 12.00-13.00 & Lect  3.1  & JG  & KCG8 \\
                                      & Mar 04   & 09.00-10.00 & Lect  3.2  & JG  & FN2  \\
                                      & Mar 06   & 12.00-13.00 & Lect  3.3  & JG  & MT4  \\
\hline
\multirow{3}{*}{\color{red}{Week 35}} & Mar 10   & 12.00-13.00 & Lect  3.4  & JG  & KCG8 \\
                                      & Mar 11   & 09.00-10.00 & Lect  3.5  & JG  & FN2  \\
                                      & Mar 13   & 12.00-13.00 & \textcolor{red}{Tutorial 5} & JG  & MT4  \\
                                      & Mar 13   & 09.00-10.00 & \textcolor{blue}{Cont. Assessment 3}& All  &  G3 St. Mary's \\
                                      & Mar 14   & 16.00-17.00 & \textcolor{red}{Tutorial 6} & JG  & 105 St Mary's (optional)\\
\hline
\multirow{3}{*}{\color{red}{Week 36}} & Mar 17   & 12.00-13.00 & Lect  4.1  & PH  & KCG8 \\
                                      & Mar 18   & 09.00-10.00 & Lect  4.2  & PH  & FN2  \\
                                      & Mar 20   & 12.00-13.00 & Lect  4.3  & PH  & MT4  \\
                                      & Mar 20   & 09.00-10.00 & \textcolor{blue}{Cont. Assessment 4}& All  &  G3 St. Mary's \\
                                      & Mar 21   & 16.00-17.00 & \textcolor{red}{Tutorial 7} & PH  & 105 St Mary's\\
                                      &          &             &             &     & NK10 \\
\hline
\multirow{3}{*}{\color{red}{Week 37}} & Mar 24   & 12.00-13.00 & Lect  4.4  & PH  & KCG8 \\
                                      & Mar 25   & 09.00-10.00 & Lect  4.5  & PH  & FN2  \\
                                      & Mar 27   & 12.00-13.00 & Lect  4.6  & PH  & MT4  \\
\hline\hline
                                      &          &             &             &     &               \\
\hline\hline
\multirow{2}{*}{\color{red}{Week 41}} & Apr 22   & 09.00-10.00 & \textcolor{red}{Tutorial 8}  & PH  & FN2  \\
                                      & Apr 24   & 12.00-13.00 & Lect  5.1  & PH  & MT4   \\
\hline
\multirow{2}{*}{\color{red}{Week 42}} & Apr 29   & 09.00-10.00 & Lect  5.2  & PH   & FN2 \\
                                      & May 01   & 12.00-13.00 & Lect  5.3  & PH  & MT4  \\
\hline
\multirow{4}{*}{\color{red}{Week 43}} & May 06   & 09.00-10.00 & \textcolor{blue}{CA Presentation 1}  & All  & FN2 \\
                                      & May 08   & 12.00-13.00 & \textcolor{blue}{CA Presentation 2}  & All & MT4 \\
                                      & May 08   & 09.00-10.00 & \textcolor{blue}{CA Presentation 3}  & All & G3 St. Mary's \\
                                      & May 09   & 16.00-17.00 & \textcolor{blue}{CA Presentation 4}  & All & 105 St Mary's \\
\hline
\multirow{5}{*}{\color{red}{Week 44}} & May 13   & 09.00-10.00 & \textcolor{red}{Tutorial 9}  & PH   & FN2  \\
                                      & May 15   & 12.00-13.00 & Revision 1  &     & MT4 \\
                                      & May 15   & 09.00-10.00 & Revision 2  &     & G3 St. Mary's \\
                                      & May 16   & 16.00-17.00 & Revision 3  &     & NK 10 \\
                                      &          &             &             &     & 105 St. Mary's \\
\hline

\hline\hline

\end{tabular}
\end{center}

\begin{center}
\begin{tabular}{||c c c c c ||}
\hline\hline
{\bf Activity} & {\bf Weeks}  & {\bf N$^{o}$ weeks} &                &       \\
\hline\hline
Lectures       &  30-37       &  8                 &   &  19h    \\
Lectures       &  41-42       &  2                 &   &  03h     \\
%Revision       &   45         &  1                 &  2 $\times$ 1h &  02h     \\
{\bf Total}    &              &                    &                &  {\bf 22h}\\
\hline
Tutorials      & 31-33, 35,  &   7               &  1h            &  {\bf 09h}  \\
               & 36, 41, 44   &                   &                &            \\
\hline
Cont. Assessment     & 32, 33, 35,  &  5                &  1h (in class) &   {\bf 08h}  \\
Activities           & 36, 43       &                   &                &      \\ 
\hline\hline
\end{tabular}
\end{center}

\section{Assessment}
\begin{enumerate}[(a)]
\item {\it First Attempt:} 1 three-hour Writtem Examination Paper (WEP, 80$\%$) + Continuous Assessment (20$\%$) + Problem Solving Exercise (01 CAS Mark, optional);
\begin{itemize}
\item CA comprises an oral presentation;
\item PSE comprises solving a computational thermodynamic problem with the following deliverables: source code (Fortran, C, C++, Python or Matlab) + oral presentation. The PSE will be awarded as {\bf either} \textcolor{red}{1} {\bf or} \textcolor{red}{0} CAS mark.
\end{itemize}
\item {\it Resit:} 1 three-hour writtem examination paper (WEP, 80$\%$) + continuous assessment mark from the {\it first attempt} (20$\%$) + Problem Solving Exercise (01 CAS Mark, optional);
\item The final CAS mark (FCM) is calculated as:
\begin{displaymath}
FCM = \min\{20, 0.8 \times WEP + 0.2 \times CA + PSE\}
\end{displaymath}
\end{enumerate}

\begin{thebibliography}{1}
%
\bibitem{smith_2001} J.M. Smith, H.C. Van Ness, M.M. Abbott (2001) {\it Introduction to Chemical Engineering Thermodynamics}, McGraw-Hill Higher Education, 6$^{th}$ Edition.
%
\bibitem{muller_2009} I. Muller, W.H. Muller (2009) {\it Fundamentals of Thermodynamics and Applications}, Springer.
%
\bibitem{cengel_2010} Y.A. \'Cengel, M.A. Boles (2010) {\it Thermodynamics -- An Engineering Approach}, McGraw-Hill Higher Education, 7$^{th}$ Edition.
%
\bibitem{devoe_2012} H. Devoe (2012) {\it Thermodynamics and Chemistry}, Free PDF Textbook, 2$^{nd}$ Edition.
%
\bibitem{rajput_2007} R.K. Rajput (2007) {\it Engineering Thermodynamics}, Laxmi Publications Ltd, 3$^{rd}$ Edition.
%
\bibitem{powers_2012} J.M. Powers (2012) {\it Lecture Notes on Thermodynamics}, Department of Aerospace and Mechanical Engineering University of Notre Dame.
%
\bibitem{borgnakke_2013} C. Borgnakke, R.E. Sonntag (2013) {\it Fundamentals of Thermodynamics}, John Wiley $\&$ Sons, 8$^{\text{th}}$ Edition.
%
\end{thebibliography}

\end{document}
