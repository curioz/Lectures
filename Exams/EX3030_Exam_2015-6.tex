
%\documentclass[calculator,steamtables,datasheet,solutions]{exam}
\documentclass[calculator,fluidstables,allquestions,datasheet]{exam}

% The full list of class options are
% calculator : Allows approved calculator use.
% datasheet : Adds a note that data sheet are attached to the exam.
% handbook : Allows the use of the engineering handbook.
% resit : Adds the resit markings to the paper.
% sample : Adds conspicuous SAMPLE markings to the paper
% solutions : Uses the contents of \solution commands (and \solmarks) to generate a solution file

\usepackage{pdfpages} 
\usepackage{lscape,comment}
 
\coursecode{EX3029}%%

\examtime{09.00--12.00}%
\examdate{03}{03}{2015}% 
\examformat{Candidates must attempt \textit{all} questions.}

\newcommand{\frc}{\displaystyle\frac}
\newcommand{\br}[1]{\!\left( #1 \right)}
\newcommand{\abs}[1]{\left| #1 \right|}
\newcommand{\fracd}[2]{\frac{\mathrm{d} #1}{\mathrm{d} #2}}
\newcommand{\fracp}[2]{\frac{\partial #1}{\partial #2}}
\renewcommand{\d}[1]{\mathrm{d} #1 } 
\newcommand{\Ma}{\mathrm{M\!a}} 

 

\begin{document}

%%%
%%% Question 01
%%%
\begin{question}
%
\begin{enumerate}[(a)]
%%% Johannes T3Q4
\item Assuming $S = S\left(P,V\right)$ and taking into consideration that,
\begin{displaymath}
\left(\frc{\partial S}{\partial T}\right)_{V} = \frc{C_{V}}{T}\;\;\;\text{ and }\;\;\; \left(\frc{\partial S}{\partial T}\right)_{P} = \frc{C_{P}}{T}
\end{displaymath}
Prove that 
\begin{displaymath}
\d S = \frc{C_{V}}{T}\left(\frc{\partial T}{\partial P}\right)_{V}\d P + \frc{C_{P}}{T}\left(\frc{\partial T}{\partial V}\right)_{P}\d V
\end{displaymath}~\marks{7}
%
\solution{hjhj}

\end{question}


\end{document}
