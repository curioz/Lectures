
%\documentclass[11pts,a4paper,amsmath,amssymb,floatfix]{article}%{report}%{book}
\documentclass[12pts,a4paper,amsmath,amssymb,floatfix]{article}%{report}%{book}
\usepackage{graphicx,wrapfig,pdfpages}% Include figure files
%\usepackage{dcolumn,enumerate}% Align table columns on decimal point
\usepackage{enumerate,enumitem}% Align table columns on decimal point
\usepackage{bm,dpfloat}% bold math
\usepackage[pdftex,bookmarks,colorlinks=true,urlcolor=rltblue,citecolor=blue]{hyperref}
\usepackage{amsfonts,amsmath,amssymb,stmaryrd,indentfirst}
\usepackage{times,psfrag}
\usepackage{natbib}
\usepackage{color}
\usepackage{units}
\usepackage{rotating}
\usepackage{multirow}
\usepackage[version=3]{mhchem}


\usepackage{pifont}
\usepackage{subfigure}
\usepackage{subeqnarray}
\usepackage{ifthen}

\usepackage{supertabular}
\usepackage{moreverb}
\usepackage{listings}
\usepackage{palatino}
%\usepackage{doi}
\usepackage{longtable}
\usepackage{float}
\usepackage{perpage}
\MakeSorted{figure}
%\usepackage{pdflscape}


%\usepackage{booktabs}
%\newcommand{\ra}[1]{\renewcommand{\arraystretch}{#1}}


\definecolor{rltblue}{rgb}{0,0,0.75}


%\usepackage{natbib}
%\usepackage{fancyhdr} %%%%
%\pagestyle{fancy}%%%%
% with this we ensure that the chapter and section
% headings are in lowercase
%%%%\renewcommand{\chaptermark}[1]{\markboth{#1}{}}
\renewcommand{\sectionmark}[1]{\markright{\thesection\ #1}}
%\fancyhf{} %delete the current section for header and footer
%\fancyhead[LE,RO]{\bfseries\thepage}
%\fancyhead[LO]{\bfseries\rightmark}
%\fancyhead[RE]{\bfseries\leftmark}
%\renewcommand{\headrulewidth}{0.5pt}
% make space for the rule
%\fancypagestyle{plain}{%
%\fancyhead{} %get rid of the headers on plain pages
%\renewcommand{\headrulewidth}{0pt} % and the line
%}

\def\newblock{\hskip .11em plus .33em minus .07em}
\usepackage{color}

%\usepackage{makeidx}
%\makeindex

\setlength\textwidth      {16.cm}
\setlength\textheight     {22.6cm}
\setlength\oddsidemargin  {-0.3cm}
\setlength\evensidemargin {0.3cm}

\setlength\headheight{14.49998pt} 
\setlength\topmargin{0.0cm}
\setlength\headsep{1.cm}
\setlength\footskip{1.cm}
\setlength\parskip{0pt}
\setlength\parindent{0pt}



\usepackage[T1]{fontenc}
\usepackage[utf8]{inputenc}
\usepackage{lmodern}
\usepackage[version=3]{mhchem}


\makeatletter
\newcounter{reaction}
%%% >> for article <<
\renewcommand\thereaction{C\,\arabic{reaction}}
%%% << for article <<
%%% >> for report and book >>
%\renewcommand\thereaction{C\,\thechapter.\arabic{reaction}}
%\@addtoreset{reaction}{chapter}
%%% << for report and book <<
\newcommand\reactiontag{\refstepcounter{reaction}\tag{\thereaction}}
\newcommand\reaction@[2][]{\begin{equation}\ce{#2}%
\ifx\@empty#1\@empty\else\label{#1}\fi%
\reactiontag\end{equation}}
\newcommand\reaction@nonumber[1]{\begin{equation*}\ce{#1}%
\end{equation*}}
\newcommand\reaction{\@ifstar{\reaction@nonumber}{\reaction@}}
\makeatother 




%%%
%%% Headers and Footers
%\lhead[] {\text{\small{EG5597 -- Advanced Chemical Engineering}}} 
%\rhead[] {{\text{\small{Tutorial 01 (2014/15)}}}}
%\chead[] {\text{\small{Session 2012/13}}} 
%\lfoot[]{Dr Jeff Gomes}
%\cfoot[\thepage]{\thepage} 
%\rfoot[\text{\small{\thepage}}]{\thepage}
%\renewcommand{\headrulewidth}{0.8pt}


%%%
%%% space between lines
%%%
\renewcommand{\baselinestretch}{1.5}

\newenvironment{VarDescription}[1]%
  {\begin{list}{}{\renewcommand{\makelabel}[1]{\textbf{##1:}\hfil}%
    \settowidth{\labelwidth}{\textbf{#1:}}%
    \setlength{\leftmargin}{\labelwidth}\addtolength{\leftmargin}{\labelsep}}}%
  {\end{list}}

%%%%%%%%%%%%%%%%%%%%%%%%%%%%%%%%%%%%%%%%%%%
%%%%%%                              %%%%%%%
%%%%%%      NOTATION SECTION        %%%%%%%
%%%%%%                              %%%%%%%
%%%%%%%%%%%%%%%%%%%%%%%%%%%%%%%%%%%%%%%%%%%

% Text abbreviations.
\newcommand{\ie}{{\em{i.e., }}}
\newcommand{\eg}{{\em{e.g., }}}
\newcommand{\wrt}{with respect to}
\newcommand{\lhs}{left hand side}
\newcommand{\rhs}{right hand side}
% Commands definining mathematical notation.

% This is for quantities which are physically vectors.
\renewcommand{\vec}[1]{{\mbox{\boldmath$#1$}}}
% Physical rank 2 tensors
\newcommand{\tensor}[1]{\overline{\overline{#1}}}
% This is for vectors formed of the value of a quantity at each node.
\newcommand{\dvec}[1]{\underline{#1}}
% This is for matrices in the discrete system.
\newcommand{\mat}[1]{\mathrm{#1}}


\DeclareMathOperator{\sgn}{sgn}
\newtheorem{thm}{Theorem}[section]
\newtheorem{lemma}[thm]{Lemma}

%\newcommand\qed{\hfill\mbox{$\Box$}}
\newcommand{\re}{{\mathrm{I}\hspace{-0.2em}\mathrm{R}}}
\newcommand{\inner}[2]{\langle#1,#2\rangle}
\renewcommand\leq{\leqslant}
\renewcommand\geq{\geqslant}
\renewcommand\le{\leqslant}
\renewcommand\ge{\geqslant}
\renewcommand\epsilon{\varepsilon}
\newcommand\eps{\varepsilon}
\renewcommand\phi{\varphi}
\newcommand{\bmF}{\vec{F}}
\newcommand{\bmphi}{\vec{\phi}}
\newcommand{\bmn}{\vec{n}}
\newcommand{\bmns}{{\textrm{\scriptsize{\boldmath $n$}}}}
\newcommand{\bmi}{\vec{i}}
\newcommand{\bmj}{\vec{j}}
\newcommand{\bmk}{\vec{k}}
\newcommand{\bmx}{\vec{x}}
\newcommand{\bmu}{\vec{u}}
\newcommand{\bmv}{\vec{v}}
\newcommand{\bmr}{\vec{r}}
\newcommand{\bma}{\vec{a}}
\newcommand{\bmg}{\vec{g}}
\newcommand{\bmU}{\vec{U}}
\newcommand{\bmI}{\vec{I}}
\newcommand{\bmq}{\vec{q}}
\newcommand{\bmT}{\vec{T}}
\newcommand{\bmM}{\vec{M}}
\newcommand{\bmtau}{\vec{\tau}}
\newcommand{\bmOmega}{\vec{\Omega}}
\newcommand{\pp}{\partial}
\newcommand{\kaptens}{\tensor{\kappa}}
\newcommand{\tautens}{\tensor{\tau}}
\newcommand{\sigtens}{\tensor{\sigma}}
\newcommand{\etens}{\tensor{\dot\epsilon}}
\newcommand{\ktens}{\tensor{k}}
\newcommand{\half}{{\textstyle \frac{1}{2}}}
\newcommand{\tote}{E}
\newcommand{\inte}{e}
\newcommand{\strt}{\dot\epsilon}
\newcommand{\modu}{|\bmu|}
% Derivatives
\renewcommand{\d}{\mathrm{d}}
\newcommand{\D}{\mathrm{D}}
\newcommand{\ddx}[2][x]{\frac{\d#2}{\d#1}}
\newcommand{\ddxx}[2][x]{\frac{\d^2#2}{\d#1^2}}
\newcommand{\ddt}[2][t]{\frac{\d#2}{\d#1}}
\newcommand{\ddtt}[2][t]{\frac{\d^2#2}{\d#1^2}}
\newcommand{\ppx}[2][x]{\frac{\partial#2}{\partial#1}}
\newcommand{\ppxx}[2][x]{\frac{\partial^2#2}{\partial#1^2}}
\newcommand{\ppt}[2][t]{\frac{\partial#2}{\partial#1}}
\newcommand{\pptt}[2][t]{\frac{\partial^2#2}{\partial#1^2}}
\newcommand{\DDx}[2][x]{\frac{\D#2}{\D#1}}
\newcommand{\DDxx}[2][x]{\frac{\D^2#2}{\D#1^2}}
\newcommand{\DDt}[2][t]{\frac{\D#2}{\D#1}}
\newcommand{\DDtt}[2][t]{\frac{\D^2#2}{\D#1^2}}
% Norms
\newcommand{\Ltwo}{\ensuremath{L_2} }
% Basis functions
\newcommand{\Qone}{\ensuremath{Q_1} }
\newcommand{\Qtwo}{\ensuremath{Q_2} }
\newcommand{\Qthree}{\ensuremath{Q_3} }
\newcommand{\QN}{\ensuremath{Q_N} }
\newcommand{\Pzero}{\ensuremath{P_0} }
\newcommand{\Pone}{\ensuremath{P_1} }
\newcommand{\Ptwo}{\ensuremath{P_2} }
\newcommand{\Pthree}{\ensuremath{P_3} }
\newcommand{\PN}{\ensuremath{P_N} }
\newcommand{\Poo}{\ensuremath{P_1P_1} }
\newcommand{\PoDGPt}{\ensuremath{P_{-1}P_2} }

\newcommand{\metric}{\tensor{M}}
\newcommand{\configureflag}[1]{\texttt{#1}}

% Units
\newcommand{\m}[1][]{\unit[#1]{m}}
\newcommand{\km}[1][]{\unit[#1]{km}}
\newcommand{\s}[1][]{\unit[#1]{s}}
\newcommand{\invs}[1][]{\unit[#1]{s}\ensuremath{^{-1}}}
\newcommand{\ms}[1][]{\unit[#1]{m\ensuremath{\,}s\ensuremath{^{-1}}}}
\newcommand{\mss}[1][]{\unit[#1]{m\ensuremath{\,}s\ensuremath{^{-2}}}}
\newcommand{\K}[1][]{\unit[#1]{K}}
\newcommand{\PSU}[1][]{\unit[#1]{PSU}}
\newcommand{\Pa}[1][]{\unit[#1]{Pa}}
\newcommand{\kg}[1][]{\unit[#1]{kg}}
\newcommand{\rads}[1][]{\unit[#1]{rad\ensuremath{\,}s\ensuremath{^{-1}}}}
\newcommand{\kgmm}[1][]{\unit[#1]{kg\ensuremath{\,}m\ensuremath{^{-2}}}}
\newcommand{\kgmmm}[1][]{\unit[#1]{kg\ensuremath{\,}m\ensuremath{^{-3}}}}
\newcommand{\Nmm}[1][]{\unit[#1]{N\ensuremath{\,}m\ensuremath{^{-2}}}}

% Dimensionless numbers
\newcommand{\dimensionless}[1]{\mathrm{#1}}
\renewcommand{\Re}{\dimensionless{Re}}
\newcommand{\Ro}{\dimensionless{Ro}}
\newcommand{\Fr}{\dimensionless{Fr}}
\newcommand{\Bu}{\dimensionless{Bu}}
\newcommand{\Ri}{\dimensionless{Ri}}
\renewcommand{\Pr}{\dimensionless{Pr}}
\newcommand{\Pe}{\dimensionless{Pe}}
\newcommand{\Ek}{\dimensionless{Ek}}
\newcommand{\Gr}{\dimensionless{Gr}}
\newcommand{\Ra}{\dimensionless{Ra}}
\newcommand{\Sh}{\dimensionless{Sh}}
\newcommand{\Sc}{\dimensionless{Sc}}


% Journals
\newcommand{\IJHMT}{{\it International Journal of Heat and Mass Transfer}}
\newcommand{\NED}{{\it Nuclear Engineering and Design}}
\newcommand{\ICHMT}{{\it International Communications in Heat and Mass Transfer}}
\newcommand{\NET}{{\it Nuclear Engineering and Technology}}
\newcommand{\HT}{{\it Heat Transfer}}   
\newcommand{\IJHT}{{\it International Journal for Heat Transfer}}

\newcommand{\frc}{\displaystyle\frac}

\newlist{ExList}{enumerate}{1}
\setlist[ExList,1]{label={\bf Example 1.} {\bf \arabic*}}

\newlist{ProbList}{enumerate}{1}
\setlist[ProbList,1]{label={\bf Problem 1.} {\bf \arabic*}}

%%%%%%%%%%%%%%%%%%%%%%%%%%%%%%%%%%%%%%%%%%%
%%%%%%                              %%%%%%%
%%%%%% END OF THE NOTATION SECTION  %%%%%%%
%%%%%%                              %%%%%%%
%%%%%%%%%%%%%%%%%%%%%%%%%%%%%%%%%%%%%%%%%%%


% Cause numbering of subsubsections. 
%\setcounter{secnumdepth}{8}
%\setcounter{tocdepth}{8}

\setcounter{secnumdepth}{4}%
\setcounter{tocdepth}{4}%

\pagenumbering{gobble}
\begin{document}

\begin{flushleft}
{\bf Question 4:}
\end{flushleft}

Calculating the velocity:
\begin{displaymath}
u = \frc{Q}{\rho \times Area} = 0.5 \text{ m.s}^{-1}
\end{displaymath}

\begin{enumerate}
\item We need to calculate the concentration profile at $t=1.5$ seconds, i.e., $j = 4$.  Using the finite difference scheme we can fill up the table:
\begin{center}
\begin{tabular}{| c | c c c c c|}
\hline 
        & $\mathcal{C}_{1}$  & $\mathcal{C}_{2}$ &  $\mathcal{C}_{3}$ &  $\mathcal{C}_{4}$  & $\mathcal{C}_{5}$ \\
\hline
$j = 1$ & 0.1500             & 1.9085           & 1.6570            &  1.0562             & 1.0000            \\
$j = 2$ & 0.1500             & 1.9472           & 1.7494            &  1.0648             & 1.1308            \\
$j = 3$ & 0.1500             & 1.9776           & 1.8547            &  1.0547             & 1.2817            \\
$j = 4$ & 0.1500             & 1.9965           & 1.9778            &  1.0198             & 1.4558            \\
\hline
\end{tabular}
\end{center}

\item The numerical solution at $t=1$ second (i.e., $j=3$) can be compared with the experimental data via $\mathcal{L}_{2}$ norm,)
\begin{displaymath}
  \text{Norm} = \frc{\left|\left|C_{i}^{\text{num}}-C_{i}^{\text{exp}}\right|\right|_{2}}{\left|\left|C_{i}^{\text{exp}}\right|\right|_{2}}
\end{displaymath} 
resulting in 0.1317. The obtained norm indicates a relatively large discrepancy between numerical and experimental data.

\end{enumerate}

\clearpage

\begin{flushleft}
{\bf Question 5:}
\end{flushleft}

\begin{enumerate}
%
\item Initial assumptions:
\begin{itemize}
\item Fluids PIB, Cl$\cdot$PIB$\cdot$Cl, MalAnh and PIBSA are incompressible liquids with densities $\rho_{PIB}$, $\rho_{ClPIBCl}$, $\rho_{MalAnh}$ and $\rho_{PIBSA}$, respectively. This can be lumped into a density of a liquid phase, i.e., $rho_{l}$
\item Fluids Cl$_{2}$ and H$_{2}$ are gasses with densities defined by equations of state (EOS). This can be lumped into a density of the gaseous phase $\rho_{g}$
\item Viscosities and thermal conductivities of all chemical species (lumped into phases) are known and represented by functions $\mu_{j}=\mu_{j}(T)$ and $\kappa_{j}=\kappa_{j}(T)$.
\item Geometry of the vessel, agitators and heat jackets are know and can be readily mapped;
\end{itemize}
%
\item Physical Formulation:
\begin{itemize}
\item Fluids PIB, Cl$_{2}$ and MalAnh are added into the domain via distinct entries, whereas PIBSA, Cl$_{2}$ and H$_{2}$ are removed through two pipelines after the reaction is finished;
\item Agitator is assumed to move in a prescribed rotation (angular momentum) -- $\mathcal{Q}_{\Omega}$;
\item The reactor is kept at a prescribed temperature through the external heat jacket;
\end{itemize}
%
\item Mathematical Formulation:
\begin{itemize}
\item Conservative equations are given;
\item Constitutive equations:
\begin{itemize}
\item fluid densities $\left(\rho_{i}\right)$: equations of state
\item fluid viscosities $\left(\mu_{i}\right)$, thermal conductivities $\left(\kappa_{i}\right)$ and heat capacities $\left(C_{p,i}\right)$: algebraic expressions;
\item empirical and semi-empirical expressions for interphase heat transfer $\left(\gamma\right)$ wall-phase heat transfer $\left(\Omega\right)$ coefficients (as function of $Nu$, $Re$ and $Pr$ dimensionless numbers);
\item differential equations representing reaction rates $\left(\mathcal{R}\right)$
\end{itemize}
\item  Initial conditions: T$_{i}\left(\underline{x},t=0\right)$, P$\left(\underline{x},t=0\right)$, $\underline{u}_{i}\left(\underline{x},t=0\right)$, $\alpha_{i}\left(\underline{x},t=0\right)$, $\omega_{i,j}\left(\underline{x},t=0\right)$.  
\item Boundary conditions:
\begin{itemize}
\item Dirichlet boundary conditions for velocity $\left(\underline{u}\right)$, temperature $\left(T_{i}\right)$, volume $\left(\alpha_{i}\right)$ and mass $\left(\omega_{i,j}\right)$ fractions in all fluid entries;
\item No flow across the walls, baffles and agitators (Robin boundary condition):
\begin{displaymath}
\left(\frc{\partial \underline{u}_{i}}{\partial \underline{n}_{j}}\right)_{w,b,a}=0
\end{displaymath}
\end{itemize}
\end{itemize}
%
\item Pre-Processing:
\begin{itemize}
\item Convert physical geometry into computational geometry (i.e., mesh generation), considering:
\begin{itemize}
\item dimensionality (2- or 3-D);
\item grid shape: triangular, quadrilateral, tetrahedral, hexahedral, prismatic etc;
\end{itemize} 
\item Discretisation method:
\begin{itemize}
\item space (FDM, FEM, FVM, etc);
\item time (explicit, implicit or hybrid methods);
\end{itemize}
\item Solver options:
\begin{itemize}
\item iterative methods;
\item direct methods as pre-conditioners;
\item tolerance;
\end{itemize}
\end{itemize}
%
\end{enumerate}

\clearpage

\begin{flushleft}
{\bf Question 4:}
\end{flushleft}
For $\alpha=20^{\circ}$C.m$^{2}$, $T_{\text{wall}}=200^{\circ}$ C, $L=0.3$ m and $T_{\text{amb}}=20^{\circ}$ C. Using central difference scheme for the second-order derivative:
\begin{displaymath}
\frc{T_{i+1}-2T_{i}+T_{i-1}}{\left(\Delta x\right)^{2}} - \alpha T_{i}= -\alpha T_{\text{amb}}\;\;\;i=1,2,\cdots
\end{displaymath}
Defining $\beta = \alpha T_{\text{amb}}$, the equation can be rearranged to
\begin{displaymath}
\frc{1}{\left(\Delta x\right)^{2}} T_{i-1}-\left[\frc{2}{\left(\Delta x\right)^{2}} +\alpha\right]T_{i} + \frc{1}{\left(\Delta x\right)^{2}}T_{i+1}=-\beta
\end{displaymath}
\begin{itemize}
\item For node $i=2$:
\begin{displaymath}
\frc{1}{\left(\Delta x\right)^{2}} T_{1}-\left[\frc{2}{\left(\Delta x\right)^{2}} +\alpha\right]T_{2} + \frc{1}{\left(\Delta x\right)^{2}}T_{3}=-\beta
\end{displaymath}
\item For node $i=3$:
\begin{displaymath}
\frc{1}{\left(\Delta x\right)^{2}} T_{2}-\left[\frc{2}{\left(\Delta x\right)^{2}} +\alpha\right]T_{3} + \frc{1}{\left(\Delta x\right)^{2}}T_{4}=-\beta
\end{displaymath}
\item $T\left(x=0\right)=T_{1}=T_{\text{wall}}$;
\item And for the second boundary condition:
\begin{displaymath}
\frc{\partial T}{\partial x}\left(x=L\right)=0
\end{displaymath}
that can be discretised with backward difference method
\begin{displaymath}
\frc{\partial T}{\partial x} = \frc{T_{i}-T_{i-1}}{\Delta x}
\end{displaymath}
where $i=4$:
\begin{displaymath}
\frc{T_{4}-T_{3}}{\Delta x} = 0 \rightarrow -\frc{1}{\Delta x}T_{3} +\frc{1}{\Delta x}T_{4} = 0 
\end{displaymath}
\item The system of equations can be written in matricial form as
\begin{displaymath}
  \begin{pmatrix}
   -\gamma                            & \frc{1}{\left(\Delta x\right)^{2}} & 0 \\ 
   \frc{1}{\left(\Delta x\right)^{2}}  & -\gamma & \frc{1}{\left(\Delta x\right)^{2}} \\
   0 & -\frc{1}{\left(\Delta x\right)^{2}} & \frc{1}{\left(\Delta x\right)^{2}}
  \end{pmatrix}
  \begin{pmatrix}
    T_{2} \\ T_{3} \\ T_{4}
  \end{pmatrix}=
  \begin{pmatrix}
   -\beta -\frc{1}{\left(\Delta x\right)^{2}}T_{1} \\
   -\beta \\
   0
  \end{pmatrix}
\end{displaymath}
where $\gamma =  \frc{2}{\left(\Delta x\right)^{2}} + \alpha$, leading to:
$T_{2}=151.71^{\circ}$C, $T_{3}=129.76^{\circ}$C and $T_{4}=129.76^{\circ}$C.
\end{itemize}

\clearpage

\begin{flushleft}
{\bf Question 5:}
\end{flushleft}
\begin{enumerate}
\item  Expanding a function $\underline{u}$ at $x_{i+1}$ about the point $x_{i}$ (assuming regular grid):
\begin{displaymath}
u\left(x_{i}+\Delta x_{i}\right) = u\left(x_{i}\right) + \Delta x_{i}\left.\frc{\partial u}{\partial x}\right|_{x_{i}} + \frc{\left(\Delta x_{i}\right)^{2}}{2!}\left.\frc{\partial^{2}u}{\partial x^{2}}\right|_{x_{i}} + \frc{\left(\Delta x_{i}\right)^{3}}{3!}\left.\frc{\partial^{3}u}{\partial x^{3}}\right|_{x_{i}} + \cdots 
\end{displaymath}
The Taylor's expansion can be rearranged as,
\begin{displaymath}
\frc{u\left(x_{i}+\Delta x_{i}\right)-u\left(x_{i}\right)}{\Delta x_{i}} - \left.\frc{\partial u}{\partial x}\right|_{x_{i}} = \frc{\Delta x_{i}}{2!}\left.\frc{\partial^{2}u}{\partial x^{2}}\right|_{x_{i}} + \frc{\left(\Delta x_{i}\right)^{2}}{3!}\left.\frc{\partial^{3}u}{\partial x^{3}}\right|_{x_{i}} + \cdots 
\end{displaymath}
The rhs of the equation is the truncation error of the series and the equation can be rewritten as
\begin{displaymath}
\left.\frc{\partial u}{\partial x}\right|_{x_{i}} = \frc{u_{i+1}-u_{i}}{\Delta x} + \mathcal{O}\left(\Delta x\right)
\end{displaymath}

\item The system of algebraic equations can be represented in matricial form as
\begin{displaymath} 
\begin{pmatrix}
2 & 0 & 1 & 0 \\
0 & 4 & 1 & 1 \\
1 & 0 & 2 & 0 \\
0 & 0 & 3 & 4 \\
\end{pmatrix}
\begin{pmatrix}
\mathcal{C}_{1} \\ \mathcal{C}_{2} \\ \mathcal{C}_{3} \\ \mathcal{C}_{4} 
\end{pmatrix}=
\begin{pmatrix}
6.35 \\ 9.00 \\ 3.95 \\ 4.11 
\end{pmatrix}
\end{displaymath}
i.e., $\underline{\mathcal{A}}\mathcal{C} = b$. 
\begin{enumerate}
\item A matrix $\mathcal{A}$ fulfill the conditions for convergence in any iterative method if any of the conditions below is true:
\begin{enumerate}
\item\label{diagonal} strictly diagonal dominant, i.e., 
\begin{displaymath}
\left|a_{ii}\right| > \sum\limits_{j=1,j\neq i}^{n} \left|a_{ij}\right|\;\; i\in\left\{1,2,\cdots,n\right\};
\end{displaymath}
\item symmetric, i.e., $\mathcal{A}^{T}=\mathcal{A}$ or;
\item positive definite, i.e., $z^{T}\mathcal{A}z>0$ for any matrix-column $z$.
\end{enumerate}
Matrix $\mathcal{A}$ satisfies condition (i) above, therefore it will converge regardless the iterative method used. 

\item In order to calculate the solution of the linear system using $\mathcal{C} = \underline{\mathcal{A}}^{-1}b$, we need to invert $\mathcal{A}$ using Gauss-Jordan method:
\begin{displaymath}
\begin{tabular}{| c c c c | c c c c |}
2 & 0 & 1 & 0 & 1 & 0 & 0 & 0 \\
0 & 4 & 1 & 1 & 0 & 1 & 0 & 0 \\
1 & 0 & 2 & 0 & 0 & 0 & 1 & 0 \\
0 & 0 & 3 & 4 & 0 & 0 & 0 & 1
\end{tabular} \Longrightarrow
\begin{tabular}{| c c c c | c c c c |}
1 & 0 & 0 & 0 & 2/3  & 0    & -1/3   & 0 \\
0 & 1 & 0 & 0 & 1/48 & 1/4  & -1/24  & -1/16 \\ 
0 & 0 & 1 & 0 & -1/3 & 0    & 2/3    & 0  \\
0 & 0 & 0 & 1 & 1/4  & 0    & -1/2   & 1/4
\end{tabular} 
\end{displaymath}
thus,
\begin{displaymath} 
\mathcal{A}^{-1}=
\begin{pmatrix}
2/3  & 0    & -1/3   & 0 \\
1/48 & 1/4  & -1/24  & -1/16 \\ 
-1/3 & 0    & 2/3    & 0  \\
1/4  & 0    & -1/2   & 1/4
\end{pmatrix}
\end{displaymath}
Now calculating $\mathcal{C} = \underline{\mathcal{A}}^{-1}b$ $\left(\text{in mg.l}^{-1}\right)$
\begin{displaymath} 
\begin{pmatrix}
\mathcal{C}_{1} \\ \mathcal{C}_{2} \\ \mathcal{C}_{3} \\ \mathcal{C}_{4} 
\end{pmatrix}=
\begin{pmatrix}
2/3  & 0    & -1/3   & 0 \\
1/48 & 1/4  & -1/24  & -1/16 \\ 
-1/3 & 0    & 2/3    & 0  \\
1/4  & 0    & -1/2   & 1/4
\end{pmatrix}
\begin{pmatrix}
6.35 \\ 9.00 \\ 3.95 \\ 4.11 
\end{pmatrix}=
\begin{pmatrix}
2.9167 \\ 1.9608 \\0.5167  \\ 0.70000 
\end{pmatrix}
\end{displaymath}

\end{enumerate}





\end{enumerate}

\end{document}
