
%\documentclass[calculator,steamtables,refrigeranttables,psychrometricchart,datasheet,solutions]{exam}
\documentclass[calculator,refrigeranttables,datasheet,resit]{exam}
% The full list of class options are
% calculator : Allows approved calculator use.
% datasheet : Adds a note that data sheet are attached to the exam.
% handbook : Allows the use of the engineering handbook.
% resit : Adds the resit markings to the paper.
% sample : Adds conspicuous SAMPLE markings to the paper
% solutions : Uses the contents of \solution commands (and \solmarks) to generate a solution file

\coursecode{EG3539}%
\coursetitle{Thermodynamics}%
\examtime{09.00--12.00}%
\examdate{16}{08}{2013}%
\examformat{Candidates must attempt \textit{all} questions.}

\newcommand{\frc}{\displaystyle\frac}

\begin{document}

%%%
%%% Question 01
%%%
\begin{question}

An engine operates in a {\it dual cycle} with compression $\left(r_{c}\right)$ and expansion $\left(r_{e}\right)$ ratios of 9 and 5, respectively. The initial pressure and temperature of the air are 1 bar and 30$^{\text{o}}$C. The heat liberated at constant pressure is twice the heat liberated at constant volume, i.e.,
\begin{displaymath}
C_{p}\left(T_{4}-T_{3}\right)=2 C_{v}\left(T_{3}-T_{2}\right).
\end{displaymath}
The isentropic expansion and compression strokes follow $PV^{n}=\text{ constant}$, with $n=1.25$. The cylinder bore (diameter) is of 250 mm and the stroke length is of 400 mm. 
\begin{enumerate}[(a)]

%%%
%%%
%%%
\item Calculate pressures and temperatures at all strokes and fill up the table below,
\begin{center}
\begin{tabular}{||c |c |c ||}
\hline \hline
 {\bf Stroke}  &  {\bf P (bar)}   &  {\bf T (K)} \\
\hline\hline
{\bf 1 }       &     1.0          &    303.15    \\
\hline         
{\bf 2 }       &                  &             \\
\hline         
{\bf 3 }       &                  &             \\
\hline         
{\bf 4 }       &                  &             \\
\hline         
{\bf 5 }       &                  &             \\
\hline\hline         
\end{tabular}
\end{center}
\begin{flushright}
{\bf [8 Marks]}
\end{flushright}
%%%
%%%
%%%
\item Sketch the {\it P-v} diagram for this cycle, indicating the swept $\left(V_{s}\right)$ and TDC $\left(V_{c}\right)$ volumes.
\begin{flushright}
{\bf [2 Marks]}
\end{flushright} 

%%%
%%%
%%%
\item Calculate the Mean Effective Pressure (MEP) of the cycle (in bar) using the following expression:
\begin{displaymath}
MEP = \frc{1}{r_{c}-1} \left[ P_{3}\left(\rho-1\right) + \frc{P_{4}\rho-P_{5}r_{c}}{n-1} - \frc{P_{2}-P_{1}r_{c}}{n-1}\right]
\end{displaymath}
\begin{flushright}
{\bf [2 Marks]}
\end{flushright}

%%%
%%%
%%%
\item Calculate {\it work done per cycle} (in kJ) and the {\it heat supplied per cycle} (in kJ), given by
\begin{displaymath}
W_{\text{cycle}} = MEP \times V_{s} \;\;\;\;{\text and } \;\;\;\; Q_{\text{cycle}} = m Q_{s} = m \left[ C_{p}\left(T_{3}-T_{2}\right) + C_{p}\left(T_{4}-T_{3}\right)\right]  
\end{displaymath}
Given: $V_{1}=V_{s}+V_{c}=\frc{r_{c}}{r_{c}-1}V_{s}$
\begin{flushright}
{\bf [2 Marks]}
\end{flushright}

%%%
%%%
%%%
\item Sketch {\it T-S} and {\it P-V} diagrams for the Otto and Diesel cycles.
\begin{flushright}
{\bf [6 Marks]}
\end{flushright} 
%
\end{enumerate}

Also given: 
\begin{eqnarray}
&& C_{p}=1.0\frc{\text{kJ}}{\text{kg.K}}\;,\;\;C_{v}=0.71\frc{\text{kJ}}{\text{kg.K}}\;, \;\; MW=29 \frc{\text{g}}{\text{gmol}}\;\; \text{(molecular weight),} \nonumber \\
&& R=8.3144621\times 10^{-5}\frc{\text{m}^{3}.\text{bar}}{\text{K.gmol}}\;\;\text{(gas constant),}\;\; \rho = \frc{r_{c}}{r_{e}} \nonumber 
\end{eqnarray}

\end{question}

\clearpage

%%%
%%% Question 2
%%%
\begin{question} \vspace{-2\baselineskip}

\begin{enumerate}[(a)]
\item A biodiesel manufacturing plant takes animal fats as feedstock and also spent cooking oil for refining. The output of the plant is 75000 tonnes per year of biodiesel.  Assigning this a calorific value of 37 MJ.kg$^{-1}$, calculate the amount of fossil fuel derived carbon dioxide eliminated by its use in preference to mineral diesel of calorific value 43 MJ.kg$^{-1}$. Given, atomic weights/g.mol$^{-1}$: C: 12 and H: 1. 
\begin{flushright}
{\bf [8 Marks]}
\end{flushright} 

\item By what chemical process can the performance of a bio-diesel be improved?
\begin{flushright}
{\bf [2 Marks]}
\end{flushright}

\item What is one major benefit of having natural gas in its liquefied form?
\begin{flushright}
{\bf [2 Marks]}
\end{flushright} 

\item Natural gas is supplied in liquefied form to a particular terminal as LNG (liquefied natural gas, heat of combustion 55 MJ.kg$^{-1}$) in a quantity of 1.2 million tonnes per annum where it is converted back to gas and used to generate electricity. At what rate will it so generate if the efficiency is 35$\%$?
\begin{flushright}
{\bf [8 Marks]}
\end{flushright} 
\end{enumerate} 

\end{question}

\clearpage

%%%
%%% Question 3
%%%
\begin{question} \vspace{-2\baselineskip}
\begin{enumerate}[(a)]

\item The steady flow energy conservation can be written in the form:
\begin{displaymath}
\frc{\dot{Q}-\dot{W}_{s}}{\dot{m}} = \left(h_{out} + \frc{u_{out}^{2}}{2}+ g z_{out}\right) - \left(h_{in} + \frc{u_{in}^{2}}{2}+ g z_{in}\right)
\end{displaymath}
\begin{enumerate}[(i)]
\item Explain the meaning of the symbols and terms in this equation.
\begin{flushright}
{\bf [3 Marks]}
\end{flushright} 
\item What assumptions are used to derive this expression?
\begin{flushright}
{\bf [3 Marks]}
\end{flushright} 
\end{enumerate}

\item The equation given above is valid for steady flow devices with one inlet and one outlet. State a modified version of this formula that is valid of steady flow devices with two inlets (whose properties are labelled 1 and 2) and one outlet (labelled 3). Give an equation representing steady mass conservation in this case.
\begin{flushright}
{\bf [3 Marks]}
\end{flushright} 

\item  A steady flow device with two inlets and one outlet does work on the fluid at a rate of 80 kW. The remaining known inlet and outlet properties are given by: \\
\begin{center}
\begin{tabular}{||l |c |c |c |c ||}
\hline\hline
{\bf Property}            & {\bf Inlet 1} & {\bf Inlet 2} & {\bf Outlet 3} & {\bf Units} \\
\hline\hline
Cross-sectional area, $A$ & 0.03          & 0.1           & 0.5            & m$^{2}$ \\
Inlet/outlet height, $z$  & 0.2           & 1.2           & 0.5            & m    \\
Volume flux, $q$          & 1.8           & 0.5           & 20             & m$^{3}$/s \\
Temperature, $T$          & 80            & 70            & 30             & $^{\text{o}}$C\\
Pressure, $p$             & 200           &               & 110            & kPa\\
\hline \hline
\end{tabular}
\end{center}
Assuming the fluid behaves as an ideal gas with  $R=0.3$ kJ/(kg.K) and $C_{p}=800$ kJ/(kg.K), calculate:
\begin{enumerate}[(i)]
\item The fluid velocity through each inlet and outlet; 
\begin{flushright}
{\bf [2 Marks]}
\end{flushright} 
\item The pressure $p_{2}$;
\begin{flushright}
{\bf [6 Marks]}
\end{flushright} 
\item The rate is heat added to gas.
\begin{flushright}
{\bf [2 Marks]}
\end{flushright} 
\end{enumerate}
\medskip

\item Comment on whether the gas transfers heat to the surroundings or whether the steady flow device heats the gas.
\begin{flushright}
{\bf [1 Mark]}
\end{flushright} 
\end{enumerate} 

\end{question}

\clearpage

%%%(saphiro 10.17)
%%% Question 4 
%%%
\begin{question} A vapour-compression refrigeration system circulates Refrigerant 134a at a rate of 6 kg/min. The refrigerant enters the compressor at -10$^{\text{o}}$C, 1.4 bar and exits at 7 bar. The isentropic compressor efficiency is 67$\%$. There are no appreciable pressure drops as the refrigerant flows through the condenser and evaporator. The refrigerant leaves the condenser at 7 bar, 24$^{o}$C. Ignoring heat transfer between the compressor and its surroundings, determine:
\begin{enumerate}[(a)]
\item $H_{i}$, $i\in\{1,2,3,4\}$;
\begin{flushright}
{\bf [8 Marks]}
\end{flushright} 
\item Coefficient of performance, $\beta$;
\begin{displaymath}
\beta = \frc{H_{1}-H_{4}}{H_{2}-H_{1}}
\end{displaymath}
\begin{flushright}
{\bf [3 Marks]}
\end{flushright} 
\item Refrigeration capacity, $\dot{Q}_{\text{in}}$ (in ton),
\begin{displaymath}
\dot{Q}_{\text{in}}=\dot{m}\left(H_{1}-H_{4}\right)
\end{displaymath}
\begin{flushright}
{\bf [3 Marks]}
\end{flushright} 
\item Sketch the schematics of the process and {\it T-S} diagram for this cycle.
\begin{flushright}
{\bf [6 Marks]}
\end{flushright} 
\end{enumerate}
In the expressions above for $\beta$ and $\dot{Q}_{\text{in}}$, assume that subscripts $1$, $2$, $3$ and $4$ represent the following flows:
\begin{itemize}
\item $1$: Evaporator $\Rightarrow$ Compressor; 
\item $2$: Compressor $\Rightarrow$ Condenser; 
\item $3$: Condenser $\Rightarrow$  Expansion Valve and; 
\item $4$: Expansion Valve $\Rightarrow$   Evaporator.
\end{itemize}
Also given: {\it 1 ton} = $1.4\times 10^{4}\;\;\frc{kJ}{h}$
\end{question}

\clearpage

%%%
%%% Question 5
%%%
\begin{question} \vspace{-2\baselineskip}
\begin{enumerate}[(a)] 
\item Define the specific humidity $\omega$, the saturation pressure of water vapour $p_{\text{v,sat}}$, and relative humidity $\varphi$. Assuming that dry air and water vapour behave like ideal gases show that
\begin{displaymath}
\omega = \frc{R_{a}\varphi p_{\text{v,sat}}}{R_{v}\left(p - \varphi p_{\text{g,v,sat}}\right)},
\end{displaymath}
where $R_{a}$ and $R_{v}$ are the specific gas constants of dry air and water vapour, respectively. 
\begin{flushright}
{\bf [5 Marks]}
\end{flushright} 

\item Air leaving an air-conditioning system in a building is mixed adiabatically with air from outside in a steady process. If the inlets to the mixing chamber are labelled 1 and 2 and the outlet is labelled 3, then state equations corresponding to the mass conservation of dry air, the mass conservation of water vapour and the conservation of energy. Hence show that
\begin{displaymath}
\frc{\dot{m}_{a2}}{\dot{m}_{a1}} = \frc{\omega_{}-\omega_{1}}{\omega_{2}-\omega_{3}} = \frc{h_{3}-h_{1}}{h_{2}-h_{3}}
\end{displaymath}
where $\dot{m}_{a}$  is a mass flux of dry air and $h$ is an enthalpy.
\begin{flushright}
{\bf [7 Marks]}
\end{flushright} 

\item Saturated air at 16$^{\text{o}}$C leaves the cooling section of an air-conditioning system in a building at a rate of 1 m$^{3}$/s. This air is mixed adiabatically at a constant pressure of 100 kPa, with air from outside that has temperature 30$^{\text{o}}$C and specific humidity 0.0182 kg H$_{2}$O/kg dry air. If the mass flux of dry air after mixing is 1.8 kg/s, then:
\begin{enumerate}[(i)]
\item Determine the mass flux through both inlets;
\begin{flushright}
{\bf [3 Marks]}
\end{flushright} 
\item Calculate the specific humidity of the air leaving the cooling section of the air-conditioning system;
\begin{flushright}
{\bf [1 Mark]}
\end{flushright} 
\item Calculate the specific humidity of the mixed air. 
\begin{flushright}
{\bf [4 Marks]}
\end{flushright} 
\end{enumerate}
\end{enumerate}
You may assume the saturation pressure of water vapour at 16$^{\text{o}}$C is 1818.747 Pa and that the specific gas constants $R_{a}=287.1$ J/(kg.K) and $R_{v}=461.5$ J/(kg.K), respectively.
\end{question}



\vfill

\paperend
\end{document}
