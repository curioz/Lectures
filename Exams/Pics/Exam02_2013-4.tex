\documentclass[calculator,steamtables,refrigeranttables]{exam}
% The full list of class options are
% calculator : Allows approved calculator use.
% datasheet : Adds a note that data sheet are attached to the exam.
% handbook : Allows the use of the engineering handbook.
% resit : Adds the resit markings to the paper.
% sample : Adds conspicuous SAMPLE markings to the paper
% solutions : Uses the contents of \solution commands (and \solmarks) to generate a solution file
% steam and refrigerant tables

\coursecode{EG3521}%
\coursetitle{Engineering Thermodynamics}%
\examtime{00.00--00.00}%
\examdate{00}{08}{2014}%
\examformat{Candidates must attempt \textit{all} questions.}

\newcommand{\frc}{\displaystyle\frac}

\begin{document}

%%%
%%% Question 01 
%%%
\begin{question} In an air-standard Otto cycle, the isentropic compression and expansion strokes are replaced by polytropic processes -- $PV^{\gamma}=\text{constant}$, with $\gamma=1.3$,
\begin{itemize}
\item {\bf 1--2:} Polytropic compression;
\item {\bf 2--3:} Addition of heat at constant volume;
\item {\bf 3--4:} Polytropic expansion and;
\item {\bf 4--1:} Rejection of heat at constant volume. 
\end{itemize}
The compression ratio $\left(\frc{V_{1}}{V_{2}}=\frc{V_{4}}{V_{3}}\right)$ is 9 for the modified cycle. At the beginning of compression, $P_{1}=1.0\;\text{bar}$ and $T_{1}=300K$. The maximum temperature during the cycle is $2000K$. 
\begin{enumerate}
\item In Table \ref{exam2_table1}, determine {\it (a)-(g)}.{\bf [7 Marks]}
\begin{table}[h]
\begin{center}
\begin{tabular}{|| c | c c c c || }
\hline\hline
        & {\bf $P$ (bar)} & {\bf $T$ (K)}  & {\bf V$\left(m^{3}/kg\right)$} & {\bf $U\;\left(kJ/kg\right)$} \\
{\bf 1} & 1               &     300        &       (a)                      & (b)                 \\    
{\bf 2} & --              &     (c)        &       --                      &  (d)                 \\
{\bf 3} & --              &    2000        &       --                       & (e)                 \\
{\bf 4} & --              &    (f)         &       --                      &  (g)                 \\
\hline\hline
\end{tabular}
\end{center}
\caption{Thermodynamic table of the modified air-standard Otto cycle (Question 1).}\label{exam2_table1}
\end{table}
%
\item Calculate the mass-based heat $\left(Q/m\right)$ and work $\left(W/m\right)$ (both in $kJ/kg$) for each stroke in the cycle; {\bf [8 Marks]}
%
\item Calculate the thermal efficiency; {\bf [3 Marks]}
\begin{displaymath}
\eta = \frc{\left(W_{\text{cycle}}/m\right)}{\left(Q_{\text{in}}/m\right)}
\end{displaymath}
\item Calculate the mean effective pressure (in bar). {\bf [2 Marks]}
\begin{displaymath}
\text{MEP} = \frc{\left(W_{\text{cycle}}/m\right)}{V_{1}-V_{2}}
\end{displaymath}
\end{enumerate} 
Also given:
\begin{itemize}
\item Isentropic relations for ideal gas:
\begin{eqnarray}
&& TV^{\gamma-1}=\text{constant} \nonumber \\
&& TP^{\frac{1-\gamma}{\gamma}}=\text{constant} \nonumber \\
&& PV^{\gamma}=\text{constant} \nonumber 
\end{eqnarray}
\item Molecular weight of air:  $MW=28.97\frc{kg}{kgmol}$;
\item Polytropic relation for an ideal gas from state $i$ to $j$:
\begin{displaymath}
\frc{W_{ij}}{m} = \int_{i}^{j}P dV = \frc{R\left(T_{j}-T_{i}\right)}{1-\gamma}
\end{displaymath}
\end{itemize}
\end{question}


\clearpage

%%%
%%% Question 2
%%%
\begin{question} \vspace{-2\baselineskip}

\end{question}

\clearpage


%%%
%%% Question 3
%%%
\begin{question}\vspace{-2\baselineskip}

\end{question}


\clearpage


%%%
%%% Question 4 
%%%
\begin{question} A vapour-compression refrigeration cycle is operated with Refrigerant R-134a as working fluid. Saturated vapour enters the compressor (with isentropic efficiency of 80$\%$) at $2$ bar, and saturated liquid exits the condenser at $8$ bar. The mass flow rate of R-134a is $7$ kg/min.
\begin{enumerate}
\item Sketch the schematic of the cycle indicating the numbering used in the calculation; {\bf [2 Marks]}
\item Calculate all enthalpies in the cycle $\left(H_{k}\;\forall k\in\left\{1,2,3,4\right\}\right)$; {\bf [8 Marks]}
\item Calculate the compressor power $\left(W_{\text{C}}\right)$ in $kW$; {\bf [2 Marks]}
\item Determine the refrigeration capacity $\left(R_{\text{n}}\right)$ in {\it tons}; {\bf [2 Marks]}
\item Calculate the coefficient of performance of the cycle $\left(\text{COP}=R_{\text{n}}/W_{\text{C}}\right)$, {\bf [2 Marks]}
\item Sketch the $TS$ diagram, indicating all stages of the cycle. {\bf [4 Marks]}
\end{enumerate}

The efficiency of the compressor can be expressed as,
\begin{displaymath}
\eta_{C}=\frc{H_{ks}-H_{k-1}}{H_{k}-H_{k-1}} 
\end{displaymath}
where $H_{ks}$ is the ideal enthalpy of the flow at stage $k$.

\end{question}

\clearpage


%%%
%%% Question 5
%%%
\begin{question} 
\begin{enumerate}%[(1)]
\item Saturated refrigerant R-134a vapour at $P_{1}=400\;kPa$ is compressed by a piston to $P_{2}=16\;\text{bar}$ in a reversible adiabatic process. Critical pressure and temperature of R-134a are 4.059 MPa and 101.06$^{\text{o}}$C.
\begin{enumerate}[(a)]
\item Calculate the work done by the piston. {\bf [6 Marks]}
\item Sketch the $TS$ and $PV$ diagrams including the constant pressure and temperature lines. {\bf [4 Marks]}
\end{enumerate}
\end{enumerate}


\end{question}



\vfill

\paperend

\end{document}
