
%\documentclass[11pts,a4paper,amsmath,amssymb,floatfix]{article}%{report}%{book}
\documentclass[12pts,a4paper,amsmath,amssymb,floatfix]{article}%{report}%{book}
\usepackage{graphicx,wrapfig,pdfpages}% Include figure files
%\usepackage{dcolumn,enumerate}% Align table columns on decimal point
\usepackage{enumerate}%,enumitem}% Align table columns on decimal point
\usepackage{bm,dpfloat}% bold math
\usepackage[pdftex,bookmarks,colorlinks=true,urlcolor=rltblue,citecolor=blue]{hyperref}
\usepackage{amsfonts,amsmath,amssymb,stmaryrd,indentfirst}
\usepackage{times,psfrag}
\usepackage{natbib}
\usepackage{color}
\usepackage{units}
\usepackage{rotating}
\usepackage{multirow}


\usepackage{pifont}
\usepackage{subfigure}
\usepackage{subeqnarray}
\usepackage{ifthen}

\usepackage{supertabular}
\usepackage{moreverb}
\usepackage{listings}
\usepackage{palatino}
%\usepackage{doi}
\usepackage{longtable}
\usepackage{float}
\usepackage{perpage}
\MakeSorted{figure}
\usepackage{lastpage}
%\usepackage{pdflscape}


%\usepackage{booktabs}
%\newcommand{\ra}[1]{\renewcommand{\arraystretch}{#1}}


\definecolor{rltblue}{rgb}{0,0,0.75}


%\usepackage{natbib}
\usepackage{fancyhdr} %%%%
\pagestyle{fancy}%%%%
% with this we ensure that the chapter and section
% headings are in lowercase
%%%%\renewcommand{\chaptermark}[1]{\markboth{#1}{}}
\renewcommand{\sectionmark}[1]{\markright{\thesection\ #1}}
\fancyhf{} %delete the current section for header and footer
\fancyhead[LE,RO]{\bfseries\thepage}
\fancyhead[LO]{\bfseries\rightmark}
\fancyhead[RE]{\bfseries\leftmark}
\renewcommand{\headrulewidth}{0.5pt}
% make space for the rule
\fancypagestyle{plain}{%
\fancyhead{} %get rid of the headers on plain pages
\renewcommand{\headrulewidth}{0pt} % and the line
}

\def\newblock{\hskip .11em plus .33em minus .07em}
\usepackage{color}

%\usepackage{makeidx}
%\makeindex

\setlength\textwidth      {16.cm}
\setlength\textheight     {22.6cm}
\setlength\oddsidemargin  {-0.3cm}
\setlength\evensidemargin {0.3cm}

\setlength\headheight{14.49998pt} 
\setlength\topmargin{0.0cm}
\setlength\headsep{1.cm}
\setlength\footskip{1.cm}
\setlength\parskip{0pt}
\setlength\parindent{0pt}


%%%
%%% Headers and Footers
\lhead[] {\text{\small{ChemEng / SafProcEng Practicals 2015/16}}} 
\rhead[] {{\text{\small{EG501V}}}}
%\chead[] {\text{\small{Session 2012/13}}} 
%\lfoot[]{Dr Jeff Gomes}
\cfoot{\thepage\ of \pageref{LastPage}}

%\cfoot[\thepage]{\thepage}
%\rfoot[\text{\small{\thepage}}]{\thepage}
\renewcommand{\headrulewidth}{0.8pt}


%%%
%%% space between lines
%%%
\renewcommand{\baselinestretch}{1.5}

\newenvironment{VarDescription}[1]%
  {\begin{list}{}{\renewcommand{\makelabel}[1]{\textbf{##1:}\hfil}%
    \settowidth{\labelwidth}{\textbf{#1:}}%
    \setlength{\leftmargin}{\labelwidth}\addtolength{\leftmargin}{\labelsep}}}%
  {\end{list}}

%%%%%%%%%%%%%%%%%%%%%%%%%%%%%%%%%%%%%%%%%%%
%%%%%%                              %%%%%%%
%%%%%%      NOTATION SECTION        %%%%%%%
%%%%%%                              %%%%%%%
%%%%%%%%%%%%%%%%%%%%%%%%%%%%%%%%%%%%%%%%%%%

% Text abbreviations.
\newcommand{\ie}{{\em{i.e., }}}
\newcommand{\eg}{{\em{e.g., }}}
\newcommand{\cf}{{\em{cf., }}}
\newcommand{\wrt}{with respect to}
\newcommand{\lhs}{left hand side}
\newcommand{\rhs}{right hand side}
% Commands definining mathematical notation.

% This is for quantities which are physically vectors.
\renewcommand{\vec}[1]{{\mbox{\boldmath$#1$}}}
% Physical rank 2 tensors
\newcommand{\tensor}[1]{\overline{\overline{#1}}}
% This is for vectors formed of the value of a quantity at each node.
\newcommand{\dvec}[1]{\underline{#1}}
% This is for matrices in the discrete system.
\newcommand{\mat}[1]{\mathrm{#1}}


\DeclareMathOperator{\sgn}{sgn}
\newtheorem{thm}{Theorem}[section]
\newtheorem{lemma}[thm]{Lemma}

%\newcommand\qed{\hfill\mbox{$\Box$}}
\newcommand{\re}{{\mathrm{I}\hspace{-0.2em}\mathrm{R}}}
\newcommand{\inner}[2]{\langle#1,#2\rangle}
\renewcommand\leq{\leqslant}
\renewcommand\geq{\geqslant}
\renewcommand\le{\leqslant}
\renewcommand\ge{\geqslant}
\renewcommand\epsilon{\varepsilon}
\newcommand\eps{\varepsilon}
\renewcommand\phi{\varphi}
\newcommand{\bmF}{\vec{F}}
\newcommand{\bmphi}{\vec{\phi}}
\newcommand{\bmn}{\vec{n}}
\newcommand{\bmns}{{\textrm{\scriptsize{\boldmath $n$}}}}
\newcommand{\bmi}{\vec{i}}
\newcommand{\bmj}{\vec{j}}
\newcommand{\bmk}{\vec{k}}
\newcommand{\bmx}{\vec{x}}
\newcommand{\bmu}{\vec{u}}
\newcommand{\bmv}{\vec{v}}
\newcommand{\bmr}{\vec{r}}
\newcommand{\bma}{\vec{a}}
\newcommand{\bmg}{\vec{g}}
\newcommand{\bmU}{\vec{U}}
\newcommand{\bmI}{\vec{I}}
\newcommand{\bmq}{\vec{q}}
\newcommand{\bmT}{\vec{T}}
\newcommand{\bmM}{\vec{M}}
\newcommand{\bmtau}{\vec{\tau}}
\newcommand{\bmOmega}{\vec{\Omega}}
\newcommand{\pp}{\partial}
\newcommand{\kaptens}{\tensor{\kappa}}
\newcommand{\tautens}{\tensor{\tau}}
\newcommand{\sigtens}{\tensor{\sigma}}
\newcommand{\etens}{\tensor{\dot\epsilon}}
\newcommand{\ktens}{\tensor{k}}
\newcommand{\half}{{\textstyle \frac{1}{2}}}
\newcommand{\tote}{E}
\newcommand{\inte}{e}
\newcommand{\strt}{\dot\epsilon}
\newcommand{\modu}{|\bmu|}
% Derivatives
\renewcommand{\d}{\mathrm{d}}
\newcommand{\D}{\mathrm{D}}
\newcommand{\ddx}[2][x]{\frac{\d#2}{\d#1}}
\newcommand{\ddxx}[2][x]{\frac{\d^2#2}{\d#1^2}}
\newcommand{\ddt}[2][t]{\frac{\d#2}{\d#1}}
\newcommand{\ddtt}[2][t]{\frac{\d^2#2}{\d#1^2}}
\newcommand{\ppx}[2][x]{\frac{\partial#2}{\partial#1}}
\newcommand{\ppxx}[2][x]{\frac{\partial^2#2}{\partial#1^2}}
\newcommand{\ppt}[2][t]{\frac{\partial#2}{\partial#1}}
\newcommand{\pptt}[2][t]{\frac{\partial^2#2}{\partial#1^2}}
\newcommand{\DDx}[2][x]{\frac{\D#2}{\D#1}}
\newcommand{\DDxx}[2][x]{\frac{\D^2#2}{\D#1^2}}
\newcommand{\DDt}[2][t]{\frac{\D#2}{\D#1}}
\newcommand{\DDtt}[2][t]{\frac{\D^2#2}{\D#1^2}}
% Norms
\newcommand{\Ltwo}{\ensuremath{L_2} }
% Basis functions
\newcommand{\Qone}{\ensuremath{Q_1} }
\newcommand{\Qtwo}{\ensuremath{Q_2} }
\newcommand{\Qthree}{\ensuremath{Q_3} }
\newcommand{\QN}{\ensuremath{Q_N} }
\newcommand{\Pzero}{\ensuremath{P_0} }
\newcommand{\Pone}{\ensuremath{P_1} }
\newcommand{\Ptwo}{\ensuremath{P_2} }
\newcommand{\Pthree}{\ensuremath{P_3} }
\newcommand{\PN}{\ensuremath{P_N} }
\newcommand{\Poo}{\ensuremath{P_1P_1} }
\newcommand{\PoDGPt}{\ensuremath{P_{-1}P_2} }

\newcommand{\metric}{\tensor{M}}
\newcommand{\configureflag}[1]{\texttt{#1}}

% Units
\newcommand{\m}[1][]{\unit[#1]{m}}
\newcommand{\km}[1][]{\unit[#1]{km}}
\newcommand{\s}[1][]{\unit[#1]{s}}
\newcommand{\invs}[1][]{\unit[#1]{s}\ensuremath{^{-1}}}
\newcommand{\ms}[1][]{\unit[#1]{m\ensuremath{\,}s\ensuremath{^{-1}}}}
\newcommand{\mss}[1][]{\unit[#1]{m\ensuremath{\,}s\ensuremath{^{-2}}}}
\newcommand{\K}[1][]{\unit[#1]{K}}
\newcommand{\PSU}[1][]{\unit[#1]{PSU}}
\newcommand{\Pa}[1][]{\unit[#1]{Pa}}
\newcommand{\kg}[1][]{\unit[#1]{kg}}
\newcommand{\rads}[1][]{\unit[#1]{rad\ensuremath{\,}s\ensuremath{^{-1}}}}
\newcommand{\kgmm}[1][]{\unit[#1]{kg\ensuremath{\,}m\ensuremath{^{-2}}}}
\newcommand{\kgmmm}[1][]{\unit[#1]{kg\ensuremath{\,}m\ensuremath{^{-3}}}}
\newcommand{\Nmm}[1][]{\unit[#1]{N\ensuremath{\,}m\ensuremath{^{-2}}}}

% Dimensionless numbers
\newcommand{\dimensionless}[1]{\mathrm{#1}}
\renewcommand{\Re}{\dimensionless{Re}}
\newcommand{\Ro}{\dimensionless{Ro}}
\newcommand{\Fr}{\dimensionless{Fr}}
\newcommand{\Bu}{\dimensionless{Bu}}
\newcommand{\Ri}{\dimensionless{Ri}}
\renewcommand{\Pr}{\dimensionless{Pr}}
\newcommand{\Pe}{\dimensionless{Pe}}
\newcommand{\Ek}{\dimensionless{Ek}}
\newcommand{\Gr}{\dimensionless{Gr}}
\newcommand{\Ra}{\dimensionless{Ra}}
\newcommand{\Sh}{\dimensionless{Sh}}
\newcommand{\Sc}{\dimensionless{Sc}}


% Journals
\newcommand{\IJHMT}{{\it International Journal of Heat and Mass Transfer}}
\newcommand{\NED}{{\it Nuclear Engineering and Design}}
\newcommand{\ICHMT}{{\it International Communications in Heat and Mass Transfer}}
\newcommand{\NET}{{\it Nuclear Engineering and Technology}}
\newcommand{\HT}{{\it Heat Transfer}}   
\newcommand{\IJHT}{{\it International Journal for Heat Transfer}}

\newcommand{\frc}{\displaystyle\frac}

%\newlist{ExList}{enumerate}{1}
%\setlist[ExList,1]{label={\bf Example 1.} {\bf \arabic*}}

%\newlist{ProbList}{enumerate}{1}
%\setlist[ProbList,1]{label={\bf Problem 1.} {\bf \arabic*}}

%%%%%%%%%%%%%%%%%%%%%%%%%%%%%%%%%%%%%%%%%%%
%%%%%%                              %%%%%%%
%%%%%% END OF THE NOTATION SECTION  %%%%%%%
%%%%%%                              %%%%%%%
%%%%%%%%%%%%%%%%%%%%%%%%%%%%%%%%%%%%%%%%%%%


% Cause numbering of subsubsections. 
%\setcounter{secnumdepth}{8}
%\setcounter{tocdepth}{8}

\setcounter{secnumdepth}{4}%
\setcounter{tocdepth}{4}%


\begin{document}

%%%
%%% FIRST PAGE
%%%
\begin{center}
{\large {\bf UNIVERSITY OF ABERDEEN, SCHOOL OF ENGINEERING}}
\medskip

{\large {\bf COURSE INFORMATION SESSION 2015/16}}
\bigskip 

{\Large {\bf EG501V Computational Fluid Dynamics}}
\end{center}

\bigskip
\begin{flushleft}

{\large {\bf CREDIT POINTS:}}\\
\hspace{0.8cm} 15
\medskip

{\large {\bf COURSE COORDINATOR: }}\\
\hspace{0.8cm} Prof Jos Derksen
\medskip 

{\large {\bf COURSE CONTRIBUTORS:}}\\
\hspace{0.8cm} Prof Dubravka Pokrajac, Dr Dominic Van der A, Dr Jefferson Gomes
\medskip

{\large {\bf DISCPLINES:}}\\
\hspace{0.8cm} MEng in Chemical Engineering and MSc in Safety Processing Engineering
\medskip  
\end{flushleft} 

\clearpage

%%%
%%% Section
%%%
\section{SYLLABUS}

\begin{enumerate}[{\bf Module 1}]
\item {\bf Introduction to CFD and Principles of Conservation:} (2 lectures)\label{Mod:Intro}
   \begin{enumerate}[(a)]
      \item Workflow for CFD design and simulation;
      \item Overview of mass, momentum and energy conservation principles;
      %\item Reynolds transport theorem;
      %\item Conservation of mass, momentum and energy;
      \item General scalar transport equation;
      %\item Navier-Stokes equation;
      \item Introduction to mesh generation technologies.
   \end{enumerate}

\begin{comment}
\item {\bf Basic Programming Concepts:} (2 lectures)\label{Mod:Programming}
  \begin{enumerate}[(a)]
       \item Source code, compilers and executables;
       \item Design and implementation of algorithms;
       \item Main programming structures: data types, operators, loops and conditionals;
       \item Arrays and arrays processing.
   \end{enumerate}
\end{comment}

\item {\bf Solution of Systems of Linear Algebraic Equations:} (3 lectures)\label{Mod:LinAlg}
   \begin{enumerate}[(a)]
      \item Review of vector-space, properties of vector and matrix norms;
      %\item Criteria for unique solution;
      \item Direct and iterative methods (Gauss elimination, LU decomposition, Jacobi, Gauss-Seidel, SOR etc);
      %\item Conjugate gradient methods;
      \item Convergence analysis.
      %\item Advanced solver techniques.
   \end{enumerate}

\item {\bf Review of Partial Differential Equations (PDE):} (2 lectures)\label{Mod:ReviewPDE}
   \begin{enumerate}[(a)]
      \item Mathematical classification of PDE (elliptic, parabolic and hyperbolic);
      \item Taylor expansion and approximate solution of PDEs;
      \item Boundary conditions.
   \end{enumerate}

\item {\bf Fundamentals of Discretisation:} (3 lectures)\label{Mod:Discretisation}
   \begin{enumerate}[(a)]
      \item Spatial and temporal discretisation principles with finite difference methods (FDM);
      \item Well-posed boundary value problem;
      \item Fundamentals of Finite Element and Finite Volume Methods (FEM and FVM);
      \item Examples of 1D steady-state heat conduction problem.
   \end{enumerate}

\item {\bf Introduction to Mesh-based Methods:} (5 lectures)\label{Mod:AdvDiffMethods}
   \begin{enumerate}[(a)]
      \item Numerical techniques for time discretisation (implicit, explicit and Crank-Nicolson schemes);
      %\item Introduction of discretisation schemes of diffusion problems;%: grid- and time-independent analysis, stability analysis, schemes for parabolic and hyperbolic equations; 
      \item Discretisation schemes for advection-diffusion problems;%: central-difference, upwind and QUICK schemes
      \item Navier-Stokes equation: discretisation, staggered / collocated grids, SIMPLE algorithm.
      %\item Introduction to high performance computing strategies for large numerical simulations.
   \end{enumerate}

\item {\bf Introduction to Turbulence Modeling:} (3 lectures)\label{Mod:Turbulence}
   \begin{enumerate}[(a)]
      \item Introduction to statistical representation of turbulent flows (homogeneous and isotropic turbulence);
      \item General properties of turbulence quantities;
      \item Reynolds-average Navier-Stokes (RANS) equation;
      \item Introduction to the main turbulence models ($\kappa-\varepsilon$, LES and DNS).
   \end{enumerate}

\item {\bf Numerical CFD Simulations (hands-on sessions):} (14 h of lab-activities)\label{Mod:Fluent}
   \begin{enumerate}[(a)]
      \item Initial training on industry standard CFD software;
      \item Mesh generation and grid quality assessment;
      \item Simulation of 2-D advection problem;
      \item Simulation of 2-/3-D steady-state and time-dependent fluid flow problems (discipline specific).
   \end{enumerate}

\end{enumerate}


\pagebreak


\section{Timetable}%\scriptsize

\begin{center}
\begin{tabular}{||c||c|c|c|c|c||}
\hline\hline
\multicolumn{5}{||c||}{Computational Fluid Dynamics (EG501V) --  Module~\ref{Mod:Fluent}} \\
\hline\hline
\multirow{5}{*}{\color{red}{Week 13}} & Oct 26    & 09.00-11.00 & Practical 01 & Edward Wright B. (Comp F81)       \\
                                      & Oct 26    & 14.00-16.00 & Lecture 01   & Auris Lecture Theatre Cruickshank \\
                                      & Oct 26    & 16.00-17.00 & Tutorial 01  & Fraser Noble (FN2)                \\
                                      & Oct 26    & 17.00-18.00 & Lecture 02   & Meston Building (MT1)             \\
                                      & Oct 30    & 16.00-18.00 & Practical 01 & MacRobert Building (Comp MR117)   \\
\hline
\multirow{5}{*}{\color{red}{Week 14}} & Nov 02    & 09.00-11.00 & Practical 02 & Edward Wright B. (Comp F81)       \\
                                      & Nov 02    & 14.00-16.00 & Lecture 03   & Auris Lecture Theatre Cruickshank \\
                                      & Nov 02    & 16.00-17.00 & Tutorial 02  & Fraser Noble (FN2)                \\
                                      & Nov 02    & 17.00-18.00 & Lecture 04  & Meston Building (MT1)              \\
                                      & Nov 06    & 16.00-18.00 & Practical 02 & MacRobert Building (Comp MR117)   \\
\hline
\multirow{2}{*}{\color{red}{Week 15}} & Nov 09    & 09.00-11.00 & Practical 03 & Edward Wright B. (Comp F81)       \\
                                      & Nov 13    & 16.00-18.00 & Practical 03 & MacRobert Building (Comp MR117)   \\
\hline
\multirow{2}{*}{\color{red}{Week 16}} & Nov 16    & 09.00-11.00 & Practical 04 & Edward Wright B. (Comp F81)       \\
                                      & Nov 20    & 16.00-18.00 & Practical 04 & MacRobert Building (Comp MR117)   \\
\hline
\multirow{2}{*}{\color{red}{Week 17}} & Nov 23    & 09.00-11.00 & Practical 05 & Edward Wright B. (Comp F81)       \\
                                      & Nov 27    & 16.00-18.00 & Practical 05 & MacRobert Building (Comp MR117)   \\
\hline\hline

\end{tabular}
\end{center}

\pagebreak

\section{Course Structure}

\begin{enumerate}[{\bf Week}]
   \item {\bf 13 $\left(\text{Oct 26}^{\text{th}}\text{/30}^{\text{th}}\right)$}: Introduction to Fluent (Fluent Tutorial Chapter 1)
      \begin{enumerate}[1]
         \item Main elements of numerical simulation (from pre- to post-processing);
         \item Introduction to mesh generation in Fluent:
             \begin{enumerate}[(a)]
                \item Main definitions: dimensionality, shape, accuracy (?) and quality;
                \item Fluent module/tool for mesh generation and linkage with engineering design tools; 
             \end{enumerate}
         \item Simple advection example -- single phase flow in a (bended) pipe:
             \begin{enumerate}[(a)]
                \item Understanding pre-processing activities: 
                   \begin{enumerate}[(i)]
                      \item Importing geometry;
                      \item Set up initial and boundary conditions and prognostic fields solvers (linkage with Modules~\ref{Mod:Discretisation}-\ref{Mod:AdvDiffMethods});
                   \end{enumerate}
                \item Understanding post-processing activities:  
                   \begin{enumerate}[(i)]
                      \item 2- and 3-D domain plots of prognostic and diagnostic fields using Fluent tool;
                      \item Time-series 2- and 3-D plots using Fluent tools;
                      \item Analysis and discussion of the results;
                   \end{enumerate}
             \end{enumerate} 
      \end{enumerate}
%
   \item {\bf 14 $\left(\text{Nov 2}^{\text{nd}}\text{/6}^{\text{th}}\right)$}: Transient Compressible Flows (Fluent Tutorial Chapter 6)
      \begin{enumerate}[1]
         \item Structure log-mesh generation:
         \item Transient flows in a nozzle:
             \begin{enumerate}[(a)]
                \item Understanding pre-processing activities: 
                   \begin{enumerate}[(i)]
                      \item Importing geometry;
                      \item Set up initial and boundary conditions and prognostic fields solvers (linkage with Modules~\ref{Mod:Discretisation}-\ref{Mod:AdvDiffMethods});
                      \item Set up turbulence submodel options (linkage with Modules~\ref{Mod:Turbulence});
                   \end{enumerate}
                \item Understanding post-processing activities:  
                   \begin{enumerate}[(i)]
                      \item 2- and 3-D domain plots of prognostic and diagnostic fields using Fluent tool;
                      \item Time-series 2- and 3-D plots using Fluent tools;
                      \item Analysis and discussion of the results;
                   \end{enumerate}
             \end{enumerate} 
      \end{enumerate}
%
   \item {\bf 15 $\left(\text{Nov 9}^{\text{nd}}\text{/13}^{\text{th}}\right)$}: Modeling Species Transport and Gaseous Combustion (Fluent Tutorial Chapter 16)
      \begin{enumerate}[1]
         \item Reaction chemistry and generation of PDF tables \textcolor{red}{(maybe?)}:
         \item Advection-diffusion-reaction problem of methane combustion in a furnace:
             \begin{enumerate}[(a)]
                \item Understanding pre-processing activities: 
                   \begin{enumerate}[(i)]
                      \item Importing geometry;
                      \item Set up initial and boundary conditions and prognostic fields solvers (linkage with Modules~\ref{Mod:Discretisation}-\ref{Mod:Turbulence});
                      \item Set up reaction tables for the species; 
                   \end{enumerate}
                \item Understanding post-processing activities:  
                   \begin{enumerate}[(i)]
                      \item 2- and 3-D domain plots of prognostic and diagnostic fields using Fluent tool;
                      \item Time-series 2- and 3-D plots using Fluent tools;
                      \item Analysis and discussion of the results;
                   \end{enumerate}
             \end{enumerate} 
      \end{enumerate}
%
   \item {\bf 16/17 $\left(\text{Nov 16}^{\text{nd}}\text{/20}^{\text{th}}\text{/23}^{\text{th}}\text{/27}^{\text{th}}\right)$}: Assignment 
      \begin{enumerate}[1]
         \item To be decided ...
      \end{enumerate}



\end{enumerate}


\end{document}
